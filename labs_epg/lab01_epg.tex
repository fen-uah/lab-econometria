% Options for packages loaded elsewhere
% Options for packages loaded elsewhere
\PassOptionsToPackage{unicode}{hyperref}
\PassOptionsToPackage{hyphens}{url}
\PassOptionsToPackage{dvipsnames,svgnames,x11names}{xcolor}
%
\documentclass[
  spanish,
  letterpaper,
  DIV=11,
  numbers=noendperiod]{scrartcl}
\usepackage{xcolor}
\usepackage{amsmath,amssymb}
\setcounter{secnumdepth}{5}
\usepackage{iftex}
\ifPDFTeX
  \usepackage[T1]{fontenc}
  \usepackage[utf8]{inputenc}
  \usepackage{textcomp} % provide euro and other symbols
\else % if luatex or xetex
  \usepackage{unicode-math} % this also loads fontspec
  \defaultfontfeatures{Scale=MatchLowercase}
  \defaultfontfeatures[\rmfamily]{Ligatures=TeX,Scale=1}
\fi
\usepackage{lmodern}
\ifPDFTeX\else
  % xetex/luatex font selection
\fi
% Use upquote if available, for straight quotes in verbatim environments
\IfFileExists{upquote.sty}{\usepackage{upquote}}{}
\IfFileExists{microtype.sty}{% use microtype if available
  \usepackage[]{microtype}
  \UseMicrotypeSet[protrusion]{basicmath} % disable protrusion for tt fonts
}{}
\makeatletter
\@ifundefined{KOMAClassName}{% if non-KOMA class
  \IfFileExists{parskip.sty}{%
    \usepackage{parskip}
  }{% else
    \setlength{\parindent}{0pt}
    \setlength{\parskip}{6pt plus 2pt minus 1pt}}
}{% if KOMA class
  \KOMAoptions{parskip=half}}
\makeatother
% Make \paragraph and \subparagraph free-standing
\makeatletter
\ifx\paragraph\undefined\else
  \let\oldparagraph\paragraph
  \renewcommand{\paragraph}{
    \@ifstar
      \xxxParagraphStar
      \xxxParagraphNoStar
  }
  \newcommand{\xxxParagraphStar}[1]{\oldparagraph*{#1}\mbox{}}
  \newcommand{\xxxParagraphNoStar}[1]{\oldparagraph{#1}\mbox{}}
\fi
\ifx\subparagraph\undefined\else
  \let\oldsubparagraph\subparagraph
  \renewcommand{\subparagraph}{
    \@ifstar
      \xxxSubParagraphStar
      \xxxSubParagraphNoStar
  }
  \newcommand{\xxxSubParagraphStar}[1]{\oldsubparagraph*{#1}\mbox{}}
  \newcommand{\xxxSubParagraphNoStar}[1]{\oldsubparagraph{#1}\mbox{}}
\fi
\makeatother

\usepackage{color}
\usepackage{fancyvrb}
\newcommand{\VerbBar}{|}
\newcommand{\VERB}{\Verb[commandchars=\\\{\}]}
\DefineVerbatimEnvironment{Highlighting}{Verbatim}{commandchars=\\\{\}}
% Add ',fontsize=\small' for more characters per line
\usepackage{framed}
\definecolor{shadecolor}{RGB}{241,243,245}
\newenvironment{Shaded}{\begin{snugshade}}{\end{snugshade}}
\newcommand{\AlertTok}[1]{\textcolor[rgb]{0.68,0.00,0.00}{#1}}
\newcommand{\AnnotationTok}[1]{\textcolor[rgb]{0.37,0.37,0.37}{#1}}
\newcommand{\AttributeTok}[1]{\textcolor[rgb]{0.40,0.45,0.13}{#1}}
\newcommand{\BaseNTok}[1]{\textcolor[rgb]{0.68,0.00,0.00}{#1}}
\newcommand{\BuiltInTok}[1]{\textcolor[rgb]{0.00,0.23,0.31}{#1}}
\newcommand{\CharTok}[1]{\textcolor[rgb]{0.13,0.47,0.30}{#1}}
\newcommand{\CommentTok}[1]{\textcolor[rgb]{0.37,0.37,0.37}{#1}}
\newcommand{\CommentVarTok}[1]{\textcolor[rgb]{0.37,0.37,0.37}{\textit{#1}}}
\newcommand{\ConstantTok}[1]{\textcolor[rgb]{0.56,0.35,0.01}{#1}}
\newcommand{\ControlFlowTok}[1]{\textcolor[rgb]{0.00,0.23,0.31}{\textbf{#1}}}
\newcommand{\DataTypeTok}[1]{\textcolor[rgb]{0.68,0.00,0.00}{#1}}
\newcommand{\DecValTok}[1]{\textcolor[rgb]{0.68,0.00,0.00}{#1}}
\newcommand{\DocumentationTok}[1]{\textcolor[rgb]{0.37,0.37,0.37}{\textit{#1}}}
\newcommand{\ErrorTok}[1]{\textcolor[rgb]{0.68,0.00,0.00}{#1}}
\newcommand{\ExtensionTok}[1]{\textcolor[rgb]{0.00,0.23,0.31}{#1}}
\newcommand{\FloatTok}[1]{\textcolor[rgb]{0.68,0.00,0.00}{#1}}
\newcommand{\FunctionTok}[1]{\textcolor[rgb]{0.28,0.35,0.67}{#1}}
\newcommand{\ImportTok}[1]{\textcolor[rgb]{0.00,0.46,0.62}{#1}}
\newcommand{\InformationTok}[1]{\textcolor[rgb]{0.37,0.37,0.37}{#1}}
\newcommand{\KeywordTok}[1]{\textcolor[rgb]{0.00,0.23,0.31}{\textbf{#1}}}
\newcommand{\NormalTok}[1]{\textcolor[rgb]{0.00,0.23,0.31}{#1}}
\newcommand{\OperatorTok}[1]{\textcolor[rgb]{0.37,0.37,0.37}{#1}}
\newcommand{\OtherTok}[1]{\textcolor[rgb]{0.00,0.23,0.31}{#1}}
\newcommand{\PreprocessorTok}[1]{\textcolor[rgb]{0.68,0.00,0.00}{#1}}
\newcommand{\RegionMarkerTok}[1]{\textcolor[rgb]{0.00,0.23,0.31}{#1}}
\newcommand{\SpecialCharTok}[1]{\textcolor[rgb]{0.37,0.37,0.37}{#1}}
\newcommand{\SpecialStringTok}[1]{\textcolor[rgb]{0.13,0.47,0.30}{#1}}
\newcommand{\StringTok}[1]{\textcolor[rgb]{0.13,0.47,0.30}{#1}}
\newcommand{\VariableTok}[1]{\textcolor[rgb]{0.07,0.07,0.07}{#1}}
\newcommand{\VerbatimStringTok}[1]{\textcolor[rgb]{0.13,0.47,0.30}{#1}}
\newcommand{\WarningTok}[1]{\textcolor[rgb]{0.37,0.37,0.37}{\textit{#1}}}

\usepackage{longtable,booktabs,array}
\usepackage{calc} % for calculating minipage widths
% Correct order of tables after \paragraph or \subparagraph
\usepackage{etoolbox}
\makeatletter
\patchcmd\longtable{\par}{\if@noskipsec\mbox{}\fi\par}{}{}
\makeatother
% Allow footnotes in longtable head/foot
\IfFileExists{footnotehyper.sty}{\usepackage{footnotehyper}}{\usepackage{footnote}}
\makesavenoteenv{longtable}
\usepackage{graphicx}
\makeatletter
\newsavebox\pandoc@box
\newcommand*\pandocbounded[1]{% scales image to fit in text height/width
  \sbox\pandoc@box{#1}%
  \Gscale@div\@tempa{\textheight}{\dimexpr\ht\pandoc@box+\dp\pandoc@box\relax}%
  \Gscale@div\@tempb{\linewidth}{\wd\pandoc@box}%
  \ifdim\@tempb\p@<\@tempa\p@\let\@tempa\@tempb\fi% select the smaller of both
  \ifdim\@tempa\p@<\p@\scalebox{\@tempa}{\usebox\pandoc@box}%
  \else\usebox{\pandoc@box}%
  \fi%
}
% Set default figure placement to htbp
\def\fps@figure{htbp}
\makeatother



\ifLuaTeX
\usepackage[bidi=basic]{babel}
\else
\usepackage[bidi=default]{babel}
\fi
% get rid of language-specific shorthands (see #6817):
\let\LanguageShortHands\languageshorthands
\def\languageshorthands#1{}


\setlength{\emergencystretch}{3em} % prevent overfull lines

\providecommand{\tightlist}{%
  \setlength{\itemsep}{0pt}\setlength{\parskip}{0pt}}



 


\KOMAoption{captions}{tableheading}
\makeatletter
\@ifpackageloaded{tcolorbox}{}{\usepackage[skins,breakable]{tcolorbox}}
\@ifpackageloaded{fontawesome5}{}{\usepackage{fontawesome5}}
\definecolor{quarto-callout-color}{HTML}{909090}
\definecolor{quarto-callout-note-color}{HTML}{0758E5}
\definecolor{quarto-callout-important-color}{HTML}{CC1914}
\definecolor{quarto-callout-warning-color}{HTML}{EB9113}
\definecolor{quarto-callout-tip-color}{HTML}{00A047}
\definecolor{quarto-callout-caution-color}{HTML}{FC5300}
\definecolor{quarto-callout-color-frame}{HTML}{acacac}
\definecolor{quarto-callout-note-color-frame}{HTML}{4582ec}
\definecolor{quarto-callout-important-color-frame}{HTML}{d9534f}
\definecolor{quarto-callout-warning-color-frame}{HTML}{f0ad4e}
\definecolor{quarto-callout-tip-color-frame}{HTML}{02b875}
\definecolor{quarto-callout-caution-color-frame}{HTML}{fd7e14}
\makeatother
\makeatletter
\@ifpackageloaded{caption}{}{\usepackage{caption}}
\AtBeginDocument{%
\ifdefined\contentsname
  \renewcommand*\contentsname{Tabla de contenidos}
\else
  \newcommand\contentsname{Tabla de contenidos}
\fi
\ifdefined\listfigurename
  \renewcommand*\listfigurename{Listado de Figuras}
\else
  \newcommand\listfigurename{Listado de Figuras}
\fi
\ifdefined\listtablename
  \renewcommand*\listtablename{Listado de Tablas}
\else
  \newcommand\listtablename{Listado de Tablas}
\fi
\ifdefined\figurename
  \renewcommand*\figurename{Figura}
\else
  \newcommand\figurename{Figura}
\fi
\ifdefined\tablename
  \renewcommand*\tablename{Tabla}
\else
  \newcommand\tablename{Tabla}
\fi
}
\@ifpackageloaded{float}{}{\usepackage{float}}
\floatstyle{ruled}
\@ifundefined{c@chapter}{\newfloat{codelisting}{h}{lop}}{\newfloat{codelisting}{h}{lop}[chapter]}
\floatname{codelisting}{Listado}
\newcommand*\listoflistings{\listof{codelisting}{Listado de Listados}}
\makeatother
\makeatletter
\makeatother
\makeatletter
\@ifpackageloaded{caption}{}{\usepackage{caption}}
\@ifpackageloaded{subcaption}{}{\usepackage{subcaption}}
\makeatother
\usepackage{bookmark}
\IfFileExists{xurl.sty}{\usepackage{xurl}}{} % add URL line breaks if available
\urlstyle{same}
\hypersetup{
  pdftitle={Capitulo 1 Introduccion\_y\_Estadistica\_Descriptiva},
  pdfauthor={Econometría para la Gestión (ECO\_EPG) - FEN UAH},
  pdflang={es},
  colorlinks=true,
  linkcolor={blue},
  filecolor={Maroon},
  citecolor={Blue},
  urlcolor={Blue},
  pdfcreator={LaTeX via pandoc}}


\title{Capitulo 1 Introduccion\_y\_Estadistica\_Descriptiva}
\author{Econometría para la Gestión (ECO\_EPG) - FEN UAH}
\date{}
\begin{document}
\maketitle

\renewcommand*\contentsname{Tabla de contenidos}
{
\hypersetup{linkcolor=}
\setcounter{tocdepth}{3}
\tableofcontents
}

\section{Material descargable}\label{material-descargable}

\href{pdf_epg/Capitulo\%20_1_Introduccion_y_Estadistica_Descriptiva.pdf}{Descargar
PDF de contenidos teóricos}

El PDF \textbf{``Capitulo 1 Introduccion\_y\_Estadistica\_Descriptiva''}
desarrolla los siguientes temas principales (a modo de índice):

\begin{itemize}
\tightlist
\item
  \textbf{1.1 Introducción}

  \begin{itemize}
  \tightlist
  \item
    Definición y términos básicos de la estadística.\\
  \item
    Diferencia entre estadística descriptiva e inferencial.\\
  \item
    Importancia de la variabilidad.
  \end{itemize}
\item
  \textbf{1.2 Elementos Fundamentales de Estadística}

  \begin{itemize}
  \tightlist
  \item
    Población, muestra, unidad experimental.\\
  \item
    Variable, parámetros poblacionales y estimadores muestrales.\\
  \item
    Ejemplos de problemas descriptivos y de inferencia.
  \end{itemize}
\item
  \textbf{1.3 Tipos de Datos}

  \begin{itemize}
  \tightlist
  \item
    Datos cuantitativos.\\
  \item
    Datos cualitativos (nominales y ordinales).\\
  \item
    Por qué el tipo de dato determina qué herramientas estadísticas
    usar.
  \end{itemize}
\item
  \textbf{1.4 Estadística Descriptiva}

  \begin{itemize}
  \tightlist
  \item
    Métodos gráficos y numéricos para describir datos cualitativos.\\
  \item
    Métodos gráficos para describir datos cuantitativos.\\
  \item
    Métodos numéricos para describir datos cuantitativos.\\
  \item
    Medidas de tendencia central (media, mediana, moda).\\
  \item
    Medidas de variación (rango, varianza, desviación estándar).\\
  \item
    Medidas de posición relativa (percentiles, cuartiles, z-scores).
  \item
    Medidas de asimetría (skewness).\\
  \item
    Medidas de concentración de datos (kurtosis).
  \end{itemize}
\end{itemize}

En este laboratorio, llevaremos varios de estos conceptos a la práctica
usando \textbf{R}.

\section{Configuración inicial en
R}\label{configuraciuxf3n-inicial-en-r}

En esta sección cargaremos las \textbf{librerías} necesarias y
definiremos la \textbf{ruta a los datos}.

\subsection{Carga de librerías}\label{carga-de-libreruxedas}

\begin{Shaded}
\begin{Highlighting}[]
\CommentTok{\# Cargamos las librerías necesarias para el laboratorio}
\FunctionTok{library}\NormalTok{(openxlsx)  }\CommentTok{\# leer archivos Excel (.xlsx)}
\FunctionTok{library}\NormalTok{(qcc)       }\CommentTok{\# diagrama de Pareto}
\FunctionTok{library}\NormalTok{(modeest)   }\CommentTok{\# moda (mfv = most frequent value)}
\FunctionTok{library}\NormalTok{(psych)     }\CommentTok{\# funciones de estadística descriptiva (útil en otros labs)}
\FunctionTok{library}\NormalTok{(moments)   }\CommentTok{\# skewness y kurtosis}
\end{Highlighting}
\end{Shaded}

\begin{tcolorbox}[enhanced jigsaw, leftrule=.75mm, opacityback=0, colframe=quarto-callout-tip-color-frame, left=2mm, rightrule=.15mm, breakable, colback=white, bottomrule=.15mm, bottomtitle=1mm, title=\textcolor{quarto-callout-tip-color}{\faLightbulb}\hspace{0.5em}{Tip}, toprule=.15mm, colbacktitle=quarto-callout-tip-color!10!white, opacitybacktitle=0.6, toptitle=1mm, coltitle=black, titlerule=0mm, arc=.35mm]

Si alguna librería no está instalada, puedes hacerlo con:

\begin{Shaded}
\begin{Highlighting}[]
\FunctionTok{install.packages}\NormalTok{(}\StringTok{"nombre\_del\_paquete"}\NormalTok{)}
\end{Highlighting}
\end{Shaded}

Por ejemplo: \texttt{install.packages("openxlsx")}.

\end{tcolorbox}

\subsection{Definir la ruta de
trabajo}\label{definir-la-ruta-de-trabajo}

Vamos a guardar la ruta donde están los datos en un objeto llamado
\texttt{ruta\_datos}.\\
Esto hace que el código sea más fácil de mantener si cambiamos de
carpeta en el futuro.

\begin{Shaded}
\begin{Highlighting}[]
\CommentTok{\# Definimos la ruta donde están los archivos de datos del laboratorio.}
\CommentTok{\# IMPORTANTE: Ajusta esta ruta si tu carpeta tiene otro nombre o ubicación.}

\NormalTok{ruta\_datos }\OtherTok{\textless{}{-}} \StringTok{"C:/Users/manue/Desktop/lab{-}econometria/labs\_epg/data\_epg"}

\CommentTok{\# Podemos verificar el contenido de la carpeta (opcional)}
\FunctionTok{list.files}\NormalTok{(ruta\_datos)}
\end{Highlighting}
\end{Shaded}

\begin{verbatim}
 [1] "annos_mantenimiento.xlsx" "auto_peso_consumo.xlsx"  
 [3] "costos.xlsx"              "data_PCA_Decathlon.csv"  
 [5] "data_PCA_ExpertWine.csv"  "Ejemplo1.xlsx"           
 [7] "Ejemplo2.xlsx"            "millaje.txt"             
 [9] "orange.csv"               "tabla_ejemplo_R.xlsx"    
\end{verbatim}

\begin{tcolorbox}[enhanced jigsaw, leftrule=.75mm, opacityback=0, colframe=quarto-callout-note-color-frame, left=2mm, rightrule=.15mm, breakable, colback=white, bottomrule=.15mm, bottomtitle=1mm, title=\textcolor{quarto-callout-note-color}{\faInfo}\hspace{0.5em}{Nota}, toprule=.15mm, colbacktitle=quarto-callout-note-color!10!white, opacitybacktitle=0.6, toptitle=1mm, coltitle=black, titlerule=0mm, arc=.35mm]

En R es recomendable usar \textbf{/ (slash)} en lugar de **** en las
rutas de Windows.\\
Por eso escribimos \texttt{"C:/Users/manue/Desktop/..."} en lugar de
\texttt{"C:Users..."}.

\end{tcolorbox}

\subsubsection{Ejemplo 1: Accidentes y métodos gráficos para datos
cualitativos}\label{ejemplo-1-accidentes-y-muxe9todos-gruxe1ficos-para-datos-cualitativos}

En este ejemplo trabajamos con \textbf{datos cualitativos} (categorías
de accidentes) y sus \textbf{frecuencias}.

La idea es:

\begin{enumerate}
\def\labelenumi{\arabic{enumi}.}
\tightlist
\item
  Leer una tabla de frecuencias desde Excel.\\
\item
  Calcular las \textbf{frecuencias relativas}.\\
\item
  Representar los datos con:

  \begin{itemize}
  \tightlist
  \item
    Gráfico de barras.\\
  \item
    Gráfico de torta (pie).\\
  \item
    Diagrama de Pareto.
  \end{itemize}
\end{enumerate}

Cargar los datos del Ejemplo 1

\begin{Shaded}
\begin{Highlighting}[]
\CommentTok{\# Construimos la ruta completa al archivo Excel del Ejemplo 1}
\NormalTok{archivo\_ejemplo1 }\OtherTok{\textless{}{-}} \FunctionTok{file.path}\NormalTok{(ruta\_datos, }\StringTok{"Ejemplo1.xlsx"}\NormalTok{)}

\CommentTok{\# Leemos el archivo Excel}
\NormalTok{datos1 }\OtherTok{\textless{}{-}} \FunctionTok{read.xlsx}\NormalTok{(}
\NormalTok{  archivo\_ejemplo1,}
  \AttributeTok{sheet    =} \StringTok{"Hoja1"}\NormalTok{,}
  \AttributeTok{colNames =} \ConstantTok{TRUE}
\NormalTok{)}

\CommentTok{\# Vemos las primeras filas para entender la estructura}
\FunctionTok{head}\NormalTok{(datos1)}
\end{Highlighting}
\end{Shaded}

\begin{verbatim}
                  Categoria Frecuencia Acumulado
1          Explosion de Gas         28        28
2 Colapso de mina de carbon          7        35
3             Falla Represa          4        39
4      Incendio Combustible          4        43
5        Descarga electrica          1        44
6           Reactor Nuclear          1        45
\end{verbatim}

\begin{tcolorbox}[enhanced jigsaw, leftrule=.75mm, opacityback=0, colframe=quarto-callout-note-color-frame, left=2mm, rightrule=.15mm, breakable, colback=white, bottomrule=.15mm, bottomtitle=1mm, title=\textcolor{quarto-callout-note-color}{\faInfo}\hspace{0.5em}{Nota}, toprule=.15mm, colbacktitle=quarto-callout-note-color!10!white, opacitybacktitle=0.6, toptitle=1mm, coltitle=black, titlerule=0mm, arc=.35mm]

En este archivo esperamos tener al menos estas columnas:

\begin{itemize}
\tightlist
\item
  \texttt{Categoria}: tipo de accidente.\\
\item
  \texttt{Frecuencia}: cuántas veces se observó cada tipo de accidente.
\end{itemize}

\end{tcolorbox}

\subsubsection{Cálculo de frecuencias
relativas}\label{cuxe1lculo-de-frecuencias-relativas}

La \textbf{frecuencia relativa} se define como:

\[\text{frecuencia relativa} = \frac{\text{frecuencia de la categoría}}{\text{total de observaciones}}. \]

\begin{Shaded}
\begin{Highlighting}[]
\CommentTok{\# Total de accidentes (suma de las frecuencias)}
\NormalTok{total }\OtherTok{\textless{}{-}} \FunctionTok{sum}\NormalTok{(datos1}\SpecialCharTok{$}\NormalTok{Frecuencia)}

\CommentTok{\# Creamos una nueva columna con la frecuencia relativa}
\NormalTok{datos1}\SpecialCharTok{$}\NormalTok{relativa }\OtherTok{\textless{}{-}}\NormalTok{ datos1}\SpecialCharTok{$}\NormalTok{Frecuencia }\SpecialCharTok{/}\NormalTok{ total}

\CommentTok{\# Revisamos la tabla con frecuencia absoluta y relativa}
\NormalTok{datos1}
\end{Highlighting}
\end{Shaded}

\begin{verbatim}
                  Categoria Frecuencia Acumulado   relativa
1          Explosion de Gas         28        28 0.62222222
2 Colapso de mina de carbon          7        35 0.15555556
3             Falla Represa          4        39 0.08888889
4      Incendio Combustible          4        43 0.08888889
5        Descarga electrica          1        44 0.02222222
6           Reactor Nuclear          1        45 0.02222222
\end{verbatim}

Ahora queremos una versión ``lista para graficar'' de las frecuencias
relativas:

\begin{Shaded}
\begin{Highlighting}[]
\CommentTok{\# Convertimos la columna relativa en un vector fila}
\NormalTok{relativo }\OtherTok{\textless{}{-}} \FunctionTok{t}\NormalTok{(}\FunctionTok{as.data.frame}\NormalTok{(datos1}\SpecialCharTok{$}\NormalTok{relativa))}

\CommentTok{\# Asignamos como nombres de columnas las categorías}
\FunctionTok{colnames}\NormalTok{(relativo) }\OtherTok{\textless{}{-}}\NormalTok{ datos1}\SpecialCharTok{$}\NormalTok{Categoria}

\NormalTok{relativo}
\end{Highlighting}
\end{Shaded}

\begin{verbatim}
                Explosion de Gas Colapso de mina de carbon Falla Represa
datos1$relativa        0.6222222                 0.1555556    0.08888889
                Incendio Combustible Descarga electrica Reactor Nuclear
datos1$relativa           0.08888889         0.02222222      0.02222222
\end{verbatim}

\begin{tcolorbox}[enhanced jigsaw, leftrule=.75mm, opacityback=0, colframe=quarto-callout-tip-color-frame, left=2mm, rightrule=.15mm, breakable, colback=white, bottomrule=.15mm, bottomtitle=1mm, title=\textcolor{quarto-callout-tip-color}{\faLightbulb}\hspace{0.5em}{Tip}, toprule=.15mm, colbacktitle=quarto-callout-tip-color!10!white, opacitybacktitle=0.6, toptitle=1mm, coltitle=black, titlerule=0mm, arc=.35mm]

\begin{itemize}
\tightlist
\item
  \texttt{t()} transpone la tabla (cambia filas por columnas).\\
\item
  Esto nos deja un \textbf{vector fila} donde cada columna es una
  categoría distinta.
\end{itemize}

\end{tcolorbox}

\subsubsection{Gráfico de barras}\label{gruxe1fico-de-barras}

El gráfico de barras es una forma estándar de mostrar
\textbf{frecuencias} (o frecuencias relativas) de variables
cualitativas.

\begin{Shaded}
\begin{Highlighting}[]
\FunctionTok{barplot}\NormalTok{(}
\NormalTok{  relativo,}
  \AttributeTok{xlab =} \StringTok{"Accidentes"}\NormalTok{,}
  \AttributeTok{ylab =} \StringTok{"Frecuencia relativa"}\NormalTok{,}
  \AttributeTok{las  =} \DecValTok{2}  \CommentTok{\# rota las etiquetas del eje X para que se lean mejor}
\NormalTok{)}
\end{Highlighting}
\end{Shaded}

\begin{center}
\pandocbounded{\includegraphics[keepaspectratio]{lab01_epg_files/figure-pdf/ejemplo1-barplot-1.pdf}}
\end{center}

\begin{tcolorbox}[enhanced jigsaw, leftrule=.75mm, opacityback=0, colframe=quarto-callout-note-color-frame, left=2mm, rightrule=.15mm, breakable, colback=white, bottomrule=.15mm, bottomtitle=1mm, title=\textcolor{quarto-callout-note-color}{\faInfo}\hspace{0.5em}{Nota}, toprule=.15mm, colbacktitle=quarto-callout-note-color!10!white, opacitybacktitle=0.6, toptitle=1mm, coltitle=black, titlerule=0mm, arc=.35mm]

\begin{itemize}
\tightlist
\item
  Cada barra representa una \textbf{categoría de accidente}.\\
\item
  La altura de la barra representa la \textbf{frecuencia relativa}
  (proporción del total).\\
\item
  \texttt{las\ =\ 2} rota las etiquetas del eje horizontal para evitar
  que se encimen.
\end{itemize}

\end{tcolorbox}

\subsubsection{Gráfico de torta (pie
chart)}\label{gruxe1fico-de-torta-pie-chart}

El gráfico de torta muestra la proporción de cada categoría como una
``porción'' de un círculo.

\begin{Shaded}
\begin{Highlighting}[]
\FunctionTok{pie}\NormalTok{(}
\NormalTok{  relativo,}
  \AttributeTok{labels =}\NormalTok{ datos1}\SpecialCharTok{$}\NormalTok{Categoria,}
  \AttributeTok{main   =} \StringTok{"Accidentes de generación de energía"}
\NormalTok{)}
\end{Highlighting}
\end{Shaded}

\begin{center}
\pandocbounded{\includegraphics[keepaspectratio]{lab01_epg_files/figure-pdf/ejemplo1-pie-1.pdf}}
\end{center}

\begin{tcolorbox}[enhanced jigsaw, leftrule=.75mm, opacityback=0, colframe=quarto-callout-tip-color-frame, left=2mm, rightrule=.15mm, breakable, colback=white, bottomrule=.15mm, bottomtitle=1mm, title=\textcolor{quarto-callout-tip-color}{\faLightbulb}\hspace{0.5em}{Tip}, toprule=.15mm, colbacktitle=quarto-callout-tip-color!10!white, opacitybacktitle=0.6, toptitle=1mm, coltitle=black, titlerule=0mm, arc=.35mm]

\begin{itemize}
\tightlist
\item
  El gráfico de torta es útil para resaltar la \textbf{proporción} de
  cada categoría.\\
\item
  Sin embargo, para comparar categorías muy parecidas entre sí, un
  \textbf{gráfico de barras} suele ser más claro.
\end{itemize}

\end{tcolorbox}

\subsubsection{Diagrama de Pareto}\label{diagrama-de-pareto}

El \textbf{diagrama de Pareto} ordena las categorías de mayor a menor
frecuencia y muestra también la acumulación.

\begin{Shaded}
\begin{Highlighting}[]
\CommentTok{\# Creamos un vector nombrado: cada elemento es la frecuencia relativa,}
\CommentTok{\# y el nombre es la categoría}
\NormalTok{relativo2 }\OtherTok{\textless{}{-}}\NormalTok{ datos1}\SpecialCharTok{$}\NormalTok{relativa}
\FunctionTok{names}\NormalTok{(relativo2) }\OtherTok{\textless{}{-}}\NormalTok{ datos1}\SpecialCharTok{$}\NormalTok{Categoria}

\CommentTok{\# Ajustamos márgenes del gráfico (opcional)}
\FunctionTok{par}\NormalTok{(}\AttributeTok{mar =} \FunctionTok{rep}\NormalTok{(}\DecValTok{4}\NormalTok{, }\DecValTok{4}\NormalTok{))}

\FunctionTok{pareto.chart}\NormalTok{(relativo2)}
\end{Highlighting}
\end{Shaded}

\begin{center}
\pandocbounded{\includegraphics[keepaspectratio]{lab01_epg_files/figure-pdf/ejemplo1-pareto-1.pdf}}
\end{center}

\begin{verbatim}
                           
Pareto chart analysis for relativo2
                               Frequency    Cum.Freq.   Percentage Cum.Percent.
  Explosion de Gas            0.62222222   0.62222222  62.22222222  62.22222222
  Colapso de mina de carbon   0.15555556   0.77777778  15.55555556  77.77777778
  Falla Represa               0.08888889   0.86666667   8.88888889  86.66666667
  Incendio Combustible        0.08888889   0.95555556   8.88888889  95.55555556
  Descarga electrica          0.02222222   0.97777778   2.22222222  97.77777778
  Reactor Nuclear             0.02222222   1.00000000   2.22222222 100.00000000
\end{verbatim}

\begin{tcolorbox}[enhanced jigsaw, leftrule=.75mm, opacityback=0, colframe=quarto-callout-note-color-frame, left=2mm, rightrule=.15mm, breakable, colback=white, bottomrule=.15mm, bottomtitle=1mm, title=\textcolor{quarto-callout-note-color}{\faInfo}\hspace{0.5em}{Nota}, toprule=.15mm, colbacktitle=quarto-callout-note-color!10!white, opacitybacktitle=0.6, toptitle=1mm, coltitle=black, titlerule=0mm, arc=.35mm]

El diagrama de Pareto permite:

\begin{itemize}
\tightlist
\item
  Identificar cuáles son las \textbf{categorías más frecuentes} (las más
  importantes).\\
\item
  Visualizar la \textbf{acumulación}: por ejemplo, ver qué porcentaje
  del total representan las primeras 2 o 3 categorías.
\end{itemize}

Esto es muy útil en gestión para aplicar el principio de
\textbf{80/20}:\\
una pequeña cantidad de causas suele explicar gran parte de los efectos.

\end{tcolorbox}

\subsubsection{Ejemplo 2: Rendimiento de vehículos y estadística
descriptiva}\label{ejemplo-2-rendimiento-de-vehuxedculos-y-estaduxedstica-descriptiva}

En este ejemplo trabajaremos con \textbf{datos cuantitativos}:
rendimiento de vehículos medido por la \textbf{EPA} (Environmental
Protection Agency).

Pasos principales:

\begin{enumerate}
\def\labelenumi{\arabic{enumi}.}
\tightlist
\item
  Convertir las unidades de rendimiento de millas/galón a km/litro.\\
\item
  Explorar los datos con:

  \begin{itemize}
  \tightlist
  \item
    Gráfico simple.\\
  \item
    Histograma.\\
  \item
    Densidad estimada y comparación con la normal.\\
  \item
    Boxplot.\\
  \end{itemize}
\item
  Calcular medidas descriptivas:

  \begin{itemize}
  \tightlist
  \item
    Media, mediana, moda, rango, desviación estándar, coeficiente de
    variación.\\
  \end{itemize}
\item
  Evaluar la normalidad:

  \begin{itemize}
  \tightlist
  \item
    Prueba de Shapiro-Wilk.\\
  \item
    Prueba de Kolmogorov-Smirnov.\\
  \item
    Gráfico QQ.\\
  \item
    Skewness y kurtosis.
  \end{itemize}
\end{enumerate}

\subsubsection{Cargar los datos del Ejemplo
2}\label{cargar-los-datos-del-ejemplo-2}

\begin{Shaded}
\begin{Highlighting}[]
\NormalTok{archivo\_ejemplo2 }\OtherTok{\textless{}{-}} \FunctionTok{file.path}\NormalTok{(ruta\_datos, }\StringTok{"Ejemplo2.xlsx"}\NormalTok{)}

\NormalTok{datos2 }\OtherTok{\textless{}{-}} \FunctionTok{read.xlsx}\NormalTok{(}
\NormalTok{  archivo\_ejemplo2,}
  \AttributeTok{sheet    =} \StringTok{"Hoja1"}\NormalTok{,}
  \AttributeTok{colNames =} \ConstantTok{TRUE}
\NormalTok{)}

\FunctionTok{head}\NormalTok{(datos2)}
\end{Highlighting}
\end{Shaded}

\begin{verbatim}
  EPA_Mileage_Ratings_milla_galon
1                            36.3
2                            32.7
3                            40.5
4                            36.2
5                            38.5
6                            36.3
\end{verbatim}

\begin{tcolorbox}[enhanced jigsaw, leftrule=.75mm, opacityback=0, colframe=quarto-callout-note-color-frame, left=2mm, rightrule=.15mm, breakable, colback=white, bottomrule=.15mm, bottomtitle=1mm, title=\textcolor{quarto-callout-note-color}{\faInfo}\hspace{0.5em}{Nota}, toprule=.15mm, colbacktitle=quarto-callout-note-color!10!white, opacitybacktitle=0.6, toptitle=1mm, coltitle=black, titlerule=0mm, arc=.35mm]

En este archivo esperamos tener una columna llamada, por ejemplo:

\begin{itemize}
\tightlist
\item
  \texttt{EPA\_Mileage\_Ratings\_milla\_galon}: rendimiento en millas
  por galón.
\end{itemize}

\end{tcolorbox}

\subsubsection{Conversión de unidades: de millas/galón a
km/litro}\label{conversiuxf3n-de-unidades-de-millasgaluxf3n-a-kmlitro}

Recordemos las equivalencias:

\begin{itemize}
\tightlist
\item
  1 milla ≈ 1.6093 kilómetros.\\
\item
  1 galón ≈ 3.78 litros.
\end{itemize}

Entonces, el factor de conversión es:

\[\text{factor} = \frac{1.6093}{3.78}.\]

\begin{Shaded}
\begin{Highlighting}[]
\NormalTok{milla }\OtherTok{\textless{}{-}} \FloatTok{1.6093}
\NormalTok{galon }\OtherTok{\textless{}{-}} \FloatTok{3.78}

\NormalTok{factorconversion }\OtherTok{\textless{}{-}}\NormalTok{ milla }\SpecialCharTok{/}\NormalTok{ galon}
\NormalTok{factorconversion}
\end{Highlighting}
\end{Shaded}

\begin{verbatim}
[1] 0.4257407
\end{verbatim}

Aplicamos este factor a la columna de rendimiento:

\begin{Shaded}
\begin{Highlighting}[]
\NormalTok{datos2}\SpecialCharTok{$}\NormalTok{EPA\_Mileage\_Ratings\_km\_l }\OtherTok{\textless{}{-}}\NormalTok{ factorconversion }\SpecialCharTok{*}\NormalTok{ datos2}\SpecialCharTok{$}\NormalTok{EPA\_Mileage\_Ratings\_milla\_galon}

\CommentTok{\# Resumen de la nueva variable}
\FunctionTok{summary}\NormalTok{(datos2}\SpecialCharTok{$}\NormalTok{EPA\_Mileage\_Ratings\_km\_l)}
\end{Highlighting}
\end{Shaded}

\begin{verbatim}
   Min. 1st Qu.  Median    Mean 3rd Qu.    Max. 
  12.77   15.19   15.75   15.75   16.32   19.12 
\end{verbatim}

\begin{tcolorbox}[enhanced jigsaw, leftrule=.75mm, opacityback=0, colframe=quarto-callout-tip-color-frame, left=2mm, rightrule=.15mm, breakable, colback=white, bottomrule=.15mm, bottomtitle=1mm, title=\textcolor{quarto-callout-tip-color}{\faLightbulb}\hspace{0.5em}{Tip}, toprule=.15mm, colbacktitle=quarto-callout-tip-color!10!white, opacitybacktitle=0.6, toptitle=1mm, coltitle=black, titlerule=0mm, arc=.35mm]

Crear una \textbf{nueva columna} (en lugar de sobrescribir la original)
permite:

\begin{itemize}
\tightlist
\item
  Conservar las unidades originales.\\
\item
  Comparar resultados y evitar errores de interpretación.
\end{itemize}

\end{tcolorbox}

\subsubsection{Gráfico simple de la
serie}\label{gruxe1fico-simple-de-la-serie}

\begin{Shaded}
\begin{Highlighting}[]
\FunctionTok{plot}\NormalTok{(}
\NormalTok{  datos2}\SpecialCharTok{$}\NormalTok{EPA\_Mileage\_Ratings\_km\_l,}
  \AttributeTok{main =} \StringTok{"Rendimiento EPA en km/l"}\NormalTok{,}
  \AttributeTok{xlab =} \StringTok{"Observación"}\NormalTok{,}
  \AttributeTok{ylab =} \StringTok{"Rendimiento (km/l)"}
\NormalTok{)}
\end{Highlighting}
\end{Shaded}

\begin{center}
\pandocbounded{\includegraphics[keepaspectratio]{lab01_epg_files/figure-pdf/ejemplo2-plot-1.pdf}}
\end{center}

Este gráfico muestra cómo varía el rendimiento (en km/l) a lo largo de
las observaciones.

\subsection{Histograma y densidad}\label{histograma-y-densidad}

\subsubsection{Histograma de frecuencias (escala de
densidad)}\label{histograma-de-frecuencias-escala-de-densidad}

\begin{Shaded}
\begin{Highlighting}[]
\FunctionTok{hist}\NormalTok{(}
\NormalTok{  datos2}\SpecialCharTok{$}\NormalTok{EPA\_Mileage\_Ratings\_km\_l,}
  \AttributeTok{main =} \StringTok{"Histograma de frecuencias EPA"}\NormalTok{,}
  \AttributeTok{xlab =} \StringTok{"Rendimiento (km/l)"}\NormalTok{,}
  \AttributeTok{freq =} \ConstantTok{FALSE}  \CommentTok{\# FALSE =\textgreater{} el eje Y representa densidad, no conteos}
\NormalTok{)}
\end{Highlighting}
\end{Shaded}

\begin{center}
\pandocbounded{\includegraphics[keepaspectratio]{lab01_epg_files/figure-pdf/ejemplo2-histograma-1.pdf}}
\end{center}

\subsubsection{Densidad estimada y comparación con la
normal}\label{densidad-estimada-y-comparaciuxf3n-con-la-normal}

\begin{Shaded}
\begin{Highlighting}[]
\CommentTok{\# Estimación de la densidad empírica}
\FunctionTok{plot}\NormalTok{(}
  \FunctionTok{density}\NormalTok{(datos2}\SpecialCharTok{$}\NormalTok{EPA\_Mileage\_Ratings\_km\_l),}
  \AttributeTok{main =} \StringTok{"Densidad de frecuencias EPA"}\NormalTok{,}
  \AttributeTok{xlab =} \StringTok{"Rendimiento (km/l)"}
\NormalTok{)}

\CommentTok{\# Añadimos la curva de una Normal con misma media y desviación estándar}
\FunctionTok{curve}\NormalTok{(}
  \FunctionTok{dnorm}\NormalTok{(}
\NormalTok{    x,}
    \AttributeTok{mean =} \FunctionTok{mean}\NormalTok{(datos2}\SpecialCharTok{$}\NormalTok{EPA\_Mileage\_Ratings\_km\_l, }\AttributeTok{na.rm =} \ConstantTok{TRUE}\NormalTok{),}
    \AttributeTok{sd   =} \FunctionTok{sd}\NormalTok{(datos2}\SpecialCharTok{$}\NormalTok{EPA\_Mileage\_Ratings\_km\_l,   }\AttributeTok{na.rm =} \ConstantTok{TRUE}\NormalTok{)}
\NormalTok{  ),}
  \AttributeTok{add =} \ConstantTok{TRUE}\NormalTok{,}
  \AttributeTok{col =} \StringTok{"red"}
\NormalTok{)}
\end{Highlighting}
\end{Shaded}

\begin{center}
\pandocbounded{\includegraphics[keepaspectratio]{lab01_epg_files/figure-pdf/ejemplo2-densidad-1.pdf}}
\end{center}

\begin{tcolorbox}[enhanced jigsaw, leftrule=.75mm, opacityback=0, colframe=quarto-callout-note-color-frame, left=2mm, rightrule=.15mm, breakable, colback=white, bottomrule=.15mm, bottomtitle=1mm, title=\textcolor{quarto-callout-note-color}{\faInfo}\hspace{0.5em}{Nota}, toprule=.15mm, colbacktitle=quarto-callout-note-color!10!white, opacitybacktitle=0.6, toptitle=1mm, coltitle=black, titlerule=0mm, arc=.35mm]

\begin{itemize}
\tightlist
\item
  La curva negra representa la \textbf{densidad empírica} estimada a
  partir de los datos.\\
\item
  La curva roja representa una \textbf{distribución normal teórica} con
  la \textbf{misma media y desviación estándar} que la muestra.\\
\item
  Comparar ambas curvas nos da una idea visual de qué tan ``normal''
  parece la distribución.
\end{itemize}

\end{tcolorbox}

\subsubsection{Boxplot (gráfico de
cajas)}\label{boxplot-gruxe1fico-de-cajas}

El boxplot resume:

\begin{itemize}
\tightlist
\item
  Mediana.\\
\item
  Cuartiles (Q1 y Q3).\\
\item
  Rango intercuartílico.\\
\item
  Posibles valores atípicos (outliers).
\end{itemize}

\begin{Shaded}
\begin{Highlighting}[]
\FunctionTok{boxplot}\NormalTok{(}
\NormalTok{  datos2}\SpecialCharTok{$}\NormalTok{EPA\_Mileage\_Ratings\_km\_l,}
  \AttributeTok{main    =} \StringTok{"Gráfico de cajas 1"}\NormalTok{,}
  \AttributeTok{ylab    =} \StringTok{"Rendimiento (km/l)"}\NormalTok{,}
  \AttributeTok{outline =} \ConstantTok{TRUE}  \CommentTok{\# muestra los posibles outliers}
\NormalTok{)}
\end{Highlighting}
\end{Shaded}

\begin{center}
\pandocbounded{\includegraphics[keepaspectratio]{lab01_epg_files/figure-pdf/ejemplo2-boxplot-1.pdf}}
\end{center}

\subsubsection{Medidas descriptivas: centro y
dispersión}\label{medidas-descriptivas-centro-y-dispersiuxf3n}

Calculamos:

\begin{itemize}
\tightlist
\item
  \textbf{Media}.\\
\item
  \textbf{Mediana}.\\
\item
  \textbf{Moda} (usando \texttt{mfv}).\\
\item
  \textbf{Rango}.\\
\item
  \textbf{Desviación estándar}.\\
\item
  \textbf{Coeficiente de variación}.
\end{itemize}

\begin{Shaded}
\begin{Highlighting}[]
\NormalTok{media   }\OtherTok{\textless{}{-}} \FunctionTok{mean}\NormalTok{(datos2}\SpecialCharTok{$}\NormalTok{EPA\_Mileage\_Ratings\_km\_l, }\AttributeTok{na.rm =} \ConstantTok{TRUE}\NormalTok{)}
\NormalTok{mediana }\OtherTok{\textless{}{-}} \FunctionTok{median}\NormalTok{(datos2}\SpecialCharTok{$}\NormalTok{EPA\_Mileage\_Ratings\_km\_l, }\AttributeTok{na.rm =} \ConstantTok{TRUE}\NormalTok{)}
\NormalTok{moda    }\OtherTok{\textless{}{-}} \FunctionTok{mfv}\NormalTok{(datos2}\SpecialCharTok{$}\NormalTok{EPA\_Mileage\_Ratings\_km\_l)  }\CommentTok{\# most frequent value}
\NormalTok{rango   }\OtherTok{\textless{}{-}} \FunctionTok{range}\NormalTok{(datos2}\SpecialCharTok{$}\NormalTok{EPA\_Mileage\_Ratings\_km\_l, }\AttributeTok{na.rm =} \ConstantTok{TRUE}\NormalTok{)}
\NormalTok{desv    }\OtherTok{\textless{}{-}} \FunctionTok{sd}\NormalTok{(datos2}\SpecialCharTok{$}\NormalTok{EPA\_Mileage\_Ratings\_km\_l, }\AttributeTok{na.rm =} \ConstantTok{TRUE}\NormalTok{)}

\NormalTok{coeficiente\_variacion }\OtherTok{\textless{}{-}}\NormalTok{ desv }\SpecialCharTok{/}\NormalTok{ media}

\NormalTok{media}
\end{Highlighting}
\end{Shaded}

\begin{verbatim}
[1] 15.74985
\end{verbatim}

\begin{Shaded}
\begin{Highlighting}[]
\NormalTok{mediana}
\end{Highlighting}
\end{Shaded}

\begin{verbatim}
[1] 15.75241
\end{verbatim}

\begin{Shaded}
\begin{Highlighting}[]
\NormalTok{moda}
\end{Highlighting}
\end{Shaded}

\begin{verbatim}
[1] 15.75241
\end{verbatim}

\begin{Shaded}
\begin{Highlighting}[]
\NormalTok{rango}
\end{Highlighting}
\end{Shaded}

\begin{verbatim}
[1] 12.77222 19.11576
\end{verbatim}

\begin{Shaded}
\begin{Highlighting}[]
\NormalTok{desv}
\end{Highlighting}
\end{Shaded}

\begin{verbatim}
[1] 1.029397
\end{verbatim}

\begin{Shaded}
\begin{Highlighting}[]
\NormalTok{coeficiente\_variacion}
\end{Highlighting}
\end{Shaded}

\begin{verbatim}
[1] 0.06535917
\end{verbatim}

\begin{tcolorbox}[enhanced jigsaw, leftrule=.75mm, opacityback=0, colframe=quarto-callout-note-color-frame, left=2mm, rightrule=.15mm, breakable, colback=white, bottomrule=.15mm, bottomtitle=1mm, title=\textcolor{quarto-callout-note-color}{\faInfo}\hspace{0.5em}{Nota}, toprule=.15mm, colbacktitle=quarto-callout-note-color!10!white, opacitybacktitle=0.6, toptitle=1mm, coltitle=black, titlerule=0mm, arc=.35mm]

\begin{itemize}
\tightlist
\item
  La \textbf{media} es el promedio aritmético.\\
\item
  La \textbf{mediana} es el valor central de los datos ordenados.\\
\item
  La \textbf{moda} es el valor que más se repite.\\
\item
  El \textbf{coeficiente de variación} es adimensional y se interpreta
  como:
\end{itemize}

\[CV = \frac{\text{desviación estándar}}{\text{media}}. \]

Es útil para comparar la variabilidad relativa entre diferentes
variables.

\end{tcolorbox}

\subsection{Pruebas de normalidad}\label{pruebas-de-normalidad}

Aplicamos dos pruebas clásicas de normalidad:

\subsubsection{Prueba de Shapiro-Wilk}\label{prueba-de-shapiro-wilk}

\begin{Shaded}
\begin{Highlighting}[]
\FunctionTok{shapiro.test}\NormalTok{(datos2}\SpecialCharTok{$}\NormalTok{EPA\_Mileage\_Ratings\_km\_l)}
\end{Highlighting}
\end{Shaded}

\begin{verbatim}

    Shapiro-Wilk normality test

data:  datos2$EPA_Mileage_Ratings_km_l
W = 0.98814, p-value = 0.5185
\end{verbatim}

\begin{itemize}
\tightlist
\item
  \textbf{H0:} los datos provienen de una distribución normal.\\
\item
  \textbf{H1:} los datos NO provienen de una distribución normal.
\end{itemize}

Si el \textbf{p-valor} es menor a 0.05, rechazamos H0 y concluimos que
los datos \textbf{no son normales}.

\subsubsection{Prueba de
Kolmogorov-Smirnov}\label{prueba-de-kolmogorov-smirnov}

\begin{Shaded}
\begin{Highlighting}[]
\FunctionTok{ks.test}\NormalTok{(}
\NormalTok{  datos2}\SpecialCharTok{$}\NormalTok{EPA\_Mileage\_Ratings\_km\_l,}
\NormalTok{  pnorm,}
  \FunctionTok{mean}\NormalTok{(datos2}\SpecialCharTok{$}\NormalTok{EPA\_Mileage\_Ratings\_km\_l, }\AttributeTok{na.rm =} \ConstantTok{TRUE}\NormalTok{),}
  \FunctionTok{sd}\NormalTok{(datos2}\SpecialCharTok{$}\NormalTok{EPA\_Mileage\_Ratings\_km\_l,   }\AttributeTok{na.rm =} \ConstantTok{TRUE}\NormalTok{)}
\NormalTok{)}
\end{Highlighting}
\end{Shaded}

\begin{verbatim}

    Asymptotic one-sample Kolmogorov-Smirnov test

data:  datos2$EPA_Mileage_Ratings_km_l
D = 0.067046, p-value = 0.7597
alternative hypothesis: two-sided
\end{verbatim}

Aquí comparamos la distribución muestral con una \textbf{normal teórica}
con:

\begin{itemize}
\tightlist
\item
  media = media muestral\\
\item
  desviación estándar = desviación estándar muestral
\end{itemize}

\subsection{Simulación de una distribución normal
comparable}\label{simulaciuxf3n-de-una-distribuciuxf3n-normal-comparable}

Generamos datos simulados desde una distribución normal con la misma
media y desviación estándar que los datos reales, y comparamos los
histogramas.

\begin{Shaded}
\begin{Highlighting}[]
\FunctionTok{set.seed}\NormalTok{(}\DecValTok{123}\NormalTok{)  }\CommentTok{\# para reproducibilidad}

\NormalTok{datos\_simulados }\OtherTok{\textless{}{-}} \FunctionTok{rnorm}\NormalTok{(}
  \DecValTok{1000}\NormalTok{,}
  \AttributeTok{mean =} \FunctionTok{mean}\NormalTok{(datos2}\SpecialCharTok{$}\NormalTok{EPA\_Mileage\_Ratings\_km\_l, }\AttributeTok{na.rm =} \ConstantTok{TRUE}\NormalTok{),}
  \AttributeTok{sd   =} \FunctionTok{sd}\NormalTok{(datos2}\SpecialCharTok{$}\NormalTok{EPA\_Mileage\_Ratings\_km\_l,   }\AttributeTok{na.rm =} \ConstantTok{TRUE}\NormalTok{)}
\NormalTok{)}

\CommentTok{\# Histograma de los datos simulados}
\FunctionTok{hist}\NormalTok{(}
\NormalTok{  datos\_simulados,}
  \AttributeTok{main =} \StringTok{"Histograma de datos simulados (Normal)"}\NormalTok{,}
  \AttributeTok{xlab =} \StringTok{"Rendimiento (km/l)"}
\NormalTok{)}
\end{Highlighting}
\end{Shaded}

\begin{center}
\pandocbounded{\includegraphics[keepaspectratio]{lab01_epg_files/figure-pdf/ejemplo2-simulacion-1.pdf}}
\end{center}

\begin{Shaded}
\begin{Highlighting}[]
\CommentTok{\# Histograma de los datos reales con curva normal}
\FunctionTok{hist}\NormalTok{(}
\NormalTok{  datos2}\SpecialCharTok{$}\NormalTok{EPA\_Mileage\_Ratings\_km\_l,}
  \AttributeTok{freq =} \ConstantTok{FALSE}\NormalTok{,}
  \AttributeTok{main =} \StringTok{"Datos EPA con curva normal teórica"}\NormalTok{,}
  \AttributeTok{xlab =} \StringTok{"Rendimiento (km/l)"}
\NormalTok{)}

\FunctionTok{curve}\NormalTok{(}
  \FunctionTok{dnorm}\NormalTok{(}
\NormalTok{    x,}
    \AttributeTok{mean =} \FunctionTok{mean}\NormalTok{(datos2}\SpecialCharTok{$}\NormalTok{EPA\_Mileage\_Ratings\_km\_l, }\AttributeTok{na.rm =} \ConstantTok{TRUE}\NormalTok{),}
    \AttributeTok{sd   =} \FunctionTok{sd}\NormalTok{(datos2}\SpecialCharTok{$}\NormalTok{EPA\_Mileage\_Ratings\_km\_l,   }\AttributeTok{na.rm =} \ConstantTok{TRUE}\NormalTok{)}
\NormalTok{  ),}
  \AttributeTok{add =} \ConstantTok{TRUE}\NormalTok{,}
  \AttributeTok{col =} \StringTok{"red"}
\NormalTok{)}
\end{Highlighting}
\end{Shaded}

\begin{center}
\pandocbounded{\includegraphics[keepaspectratio]{lab01_epg_files/figure-pdf/ejemplo2-simulacion-2.pdf}}
\end{center}

\subsection{Cuantiles y valores
teóricos}\label{cuantiles-y-valores-teuxf3ricos}

Calculamos los \textbf{cuantiles} (percentiles) y luego los valores
teóricos de una normal para los percentiles 25\% y 75\% usando
\texttt{qnorm}.

\begin{Shaded}
\begin{Highlighting}[]
\CommentTok{\# Cuantiles 0\%, 25\%, 50\%, 75\% y 100\%}
\FunctionTok{quantile}\NormalTok{(}
\NormalTok{  datos2}\SpecialCharTok{$}\NormalTok{EPA\_Mileage\_Ratings\_km\_l,}
  \AttributeTok{prob =} \FunctionTok{c}\NormalTok{(}\DecValTok{0}\NormalTok{, }\FloatTok{0.25}\NormalTok{, }\FloatTok{0.5}\NormalTok{, }\FloatTok{0.75}\NormalTok{, }\DecValTok{1}\NormalTok{),}
  \AttributeTok{na.rm =} \ConstantTok{TRUE}
\NormalTok{)}
\end{Highlighting}
\end{Shaded}

\begin{verbatim}
      0%      25%      50%      75%     100% 
12.77222 15.18830 15.75241 16.31651 19.11576 
\end{verbatim}

Ahora calculamos los valores que corresponderían al 25\% y 75\% bajo una
distribución normal con la misma media y desviación estándar:

\begin{Shaded}
\begin{Highlighting}[]
\NormalTok{z\_0}\FloatTok{.75} \OtherTok{\textless{}{-}} \FunctionTok{mean}\NormalTok{(datos2}\SpecialCharTok{$}\NormalTok{EPA\_Mileage\_Ratings\_km\_l, }\AttributeTok{na.rm =} \ConstantTok{TRUE}\NormalTok{) }\SpecialCharTok{+}
  \FunctionTok{qnorm}\NormalTok{(}\FloatTok{0.75}\NormalTok{) }\SpecialCharTok{*} \FunctionTok{sd}\NormalTok{(datos2}\SpecialCharTok{$}\NormalTok{EPA\_Mileage\_Ratings\_km\_l, }\AttributeTok{na.rm =} \ConstantTok{TRUE}\NormalTok{)}

\NormalTok{z\_0}\FloatTok{.25} \OtherTok{\textless{}{-}} \FunctionTok{mean}\NormalTok{(datos2}\SpecialCharTok{$}\NormalTok{EPA\_Mileage\_Ratings\_km\_l, }\AttributeTok{na.rm =} \ConstantTok{TRUE}\NormalTok{) }\SpecialCharTok{+}
  \FunctionTok{qnorm}\NormalTok{(}\FloatTok{0.25}\NormalTok{) }\SpecialCharTok{*} \FunctionTok{sd}\NormalTok{(datos2}\SpecialCharTok{$}\NormalTok{EPA\_Mileage\_Ratings\_km\_l, }\AttributeTok{na.rm =} \ConstantTok{TRUE}\NormalTok{)}

\NormalTok{z\_0}\FloatTok{.75}
\end{Highlighting}
\end{Shaded}

\begin{verbatim}
[1] 16.44417
\end{verbatim}

\begin{Shaded}
\begin{Highlighting}[]
\NormalTok{z\_0}\FloatTok{.25}
\end{Highlighting}
\end{Shaded}

\begin{verbatim}
[1] 15.05554
\end{verbatim}

\subsection{Gráfico QQ
(quantile-quantile)}\label{gruxe1fico-qq-quantile-quantile}

El gráfico QQ compara los cuantiles de los datos con los cuantiles de
una normal teórica.

\begin{Shaded}
\begin{Highlighting}[]
\FunctionTok{qqnorm}\NormalTok{(datos2}\SpecialCharTok{$}\NormalTok{EPA\_Mileage\_Ratings\_km\_l)}
\FunctionTok{qqline}\NormalTok{(datos2}\SpecialCharTok{$}\NormalTok{EPA\_Mileage\_Ratings\_km\_l, }\AttributeTok{col =} \StringTok{"red"}\NormalTok{)}
\end{Highlighting}
\end{Shaded}

\begin{center}
\pandocbounded{\includegraphics[keepaspectratio]{lab01_epg_files/figure-pdf/ejemplo2-qqplot-1.pdf}}
\end{center}

\begin{itemize}
\tightlist
\item
  Si los puntos siguen aproximadamente la línea roja, la distribución de
  los datos es \textbf{cercana a la normal}.\\
\item
  Desviaciones sistemáticas indican diferencias respecto a la normalidad
  (colas más pesadas, asimetría, etc.).
\end{itemize}

\subsection{Asimetría (skewness) y curtosis
(kurtosis)}\label{asimetruxeda-skewness-y-curtosis-kurtosis}

Finalmente, medimos la \textbf{forma} de la distribución mediante:

\begin{itemize}
\tightlist
\item
  \textbf{Skewness}: mide la asimetría.\\
\item
  \textbf{Kurtosis}: mide el ``apuntamiento'' o peso de las colas
  comparado con una normal.
\end{itemize}

\begin{Shaded}
\begin{Highlighting}[]
\FunctionTok{skewness}\NormalTok{(datos2}\SpecialCharTok{$}\NormalTok{EPA\_Mileage\_Ratings\_km\_l, }\AttributeTok{na.rm =} \ConstantTok{TRUE}\NormalTok{)}
\end{Highlighting}
\end{Shaded}

\begin{verbatim}
[1] 0.05014194
\end{verbatim}

\begin{Shaded}
\begin{Highlighting}[]
\FunctionTok{kurtosis}\NormalTok{(datos2}\SpecialCharTok{$}\NormalTok{EPA\_Mileage\_Ratings\_km\_l, }\AttributeTok{na.rm =} \ConstantTok{TRUE}\NormalTok{)}
\end{Highlighting}
\end{Shaded}

\begin{verbatim}
[1] 3.672556
\end{verbatim}

\begin{tcolorbox}[enhanced jigsaw, leftrule=.75mm, opacityback=0, colframe=quarto-callout-note-color-frame, left=2mm, rightrule=.15mm, breakable, colback=white, bottomrule=.15mm, bottomtitle=1mm, title=\textcolor{quarto-callout-note-color}{\faInfo}\hspace{0.5em}{Nota}, toprule=.15mm, colbacktitle=quarto-callout-note-color!10!white, opacitybacktitle=0.6, toptitle=1mm, coltitle=black, titlerule=0mm, arc=.35mm]

\begin{itemize}
\item
  Si \textbf{skewness \textgreater{} 0}: asimetría hacia la derecha
  (cola más larga a la derecha).\\
\item
  Si \textbf{skewness \textless{} 0}: asimetría hacia la izquierda.
\item
  Para la curtosis (definición clásica):

  \begin{itemize}
  \tightlist
  \item
    Una distribución normal tiene \textbf{kurtosis ≈ 3}.\\
  \item
    Si kurtosis \textgreater{} 3: colas más pesadas (leptocúrtica).\\
  \item
    Si kurtosis \textless{} 3: colas más ligeras (platicúrtica).
  \end{itemize}
\end{itemize}

\end{tcolorbox}

\subsection{Resumen final del Ejemplo
2}\label{resumen-final-del-ejemplo-2}

En este ejemplo aprendimos a:

\begin{itemize}
\tightlist
\item
  Convertir unidades (millas/galón → km/l).\\
\item
  Explorar datos cuantitativos con gráficos:

  \begin{itemize}
  \tightlist
  \item
    Plot simple, histograma, densidad, boxplot y QQ-plot.\\
  \end{itemize}
\item
  Calcular medidas descriptivas:

  \begin{itemize}
  \tightlist
  \item
    Media, mediana, moda, rango, desviación estándar, coeficiente de
    variación.\\
  \end{itemize}
\item
  Evaluar la normalidad de una variable usando:

  \begin{itemize}
  \tightlist
  \item
    Pruebas de hipótesis (Shapiro-Wilk, Kolmogorov-Smirnov).\\
  \item
    Comparación visual con una distribución normal.\\
  \item
    Skewness y kurtosis.
  \end{itemize}
\end{itemize}

Estos pasos son la base para análisis más avanzados en econometría,
donde \textbf{suponemos frecuentemente normalidad} en los errores o en
ciertas variables.




\end{document}
