% Options for packages loaded elsewhere
% Options for packages loaded elsewhere
\PassOptionsToPackage{unicode}{hyperref}
\PassOptionsToPackage{hyphens}{url}
\PassOptionsToPackage{dvipsnames,svgnames,x11names}{xcolor}
%
\documentclass[
  spanish,
  letterpaper,
  DIV=11,
  numbers=noendperiod]{scrartcl}
\usepackage{xcolor}
\usepackage{amsmath,amssymb}
\setcounter{secnumdepth}{5}
\usepackage{iftex}
\ifPDFTeX
  \usepackage[T1]{fontenc}
  \usepackage[utf8]{inputenc}
  \usepackage{textcomp} % provide euro and other symbols
\else % if luatex or xetex
  \usepackage{unicode-math} % this also loads fontspec
  \defaultfontfeatures{Scale=MatchLowercase}
  \defaultfontfeatures[\rmfamily]{Ligatures=TeX,Scale=1}
\fi
\usepackage{lmodern}
\ifPDFTeX\else
  % xetex/luatex font selection
\fi
% Use upquote if available, for straight quotes in verbatim environments
\IfFileExists{upquote.sty}{\usepackage{upquote}}{}
\IfFileExists{microtype.sty}{% use microtype if available
  \usepackage[]{microtype}
  \UseMicrotypeSet[protrusion]{basicmath} % disable protrusion for tt fonts
}{}
\makeatletter
\@ifundefined{KOMAClassName}{% if non-KOMA class
  \IfFileExists{parskip.sty}{%
    \usepackage{parskip}
  }{% else
    \setlength{\parindent}{0pt}
    \setlength{\parskip}{6pt plus 2pt minus 1pt}}
}{% if KOMA class
  \KOMAoptions{parskip=half}}
\makeatother
% Make \paragraph and \subparagraph free-standing
\makeatletter
\ifx\paragraph\undefined\else
  \let\oldparagraph\paragraph
  \renewcommand{\paragraph}{
    \@ifstar
      \xxxParagraphStar
      \xxxParagraphNoStar
  }
  \newcommand{\xxxParagraphStar}[1]{\oldparagraph*{#1}\mbox{}}
  \newcommand{\xxxParagraphNoStar}[1]{\oldparagraph{#1}\mbox{}}
\fi
\ifx\subparagraph\undefined\else
  \let\oldsubparagraph\subparagraph
  \renewcommand{\subparagraph}{
    \@ifstar
      \xxxSubParagraphStar
      \xxxSubParagraphNoStar
  }
  \newcommand{\xxxSubParagraphStar}[1]{\oldsubparagraph*{#1}\mbox{}}
  \newcommand{\xxxSubParagraphNoStar}[1]{\oldsubparagraph{#1}\mbox{}}
\fi
\makeatother

\usepackage{color}
\usepackage{fancyvrb}
\newcommand{\VerbBar}{|}
\newcommand{\VERB}{\Verb[commandchars=\\\{\}]}
\DefineVerbatimEnvironment{Highlighting}{Verbatim}{commandchars=\\\{\}}
% Add ',fontsize=\small' for more characters per line
\usepackage{framed}
\definecolor{shadecolor}{RGB}{241,243,245}
\newenvironment{Shaded}{\begin{snugshade}}{\end{snugshade}}
\newcommand{\AlertTok}[1]{\textcolor[rgb]{0.68,0.00,0.00}{#1}}
\newcommand{\AnnotationTok}[1]{\textcolor[rgb]{0.37,0.37,0.37}{#1}}
\newcommand{\AttributeTok}[1]{\textcolor[rgb]{0.40,0.45,0.13}{#1}}
\newcommand{\BaseNTok}[1]{\textcolor[rgb]{0.68,0.00,0.00}{#1}}
\newcommand{\BuiltInTok}[1]{\textcolor[rgb]{0.00,0.23,0.31}{#1}}
\newcommand{\CharTok}[1]{\textcolor[rgb]{0.13,0.47,0.30}{#1}}
\newcommand{\CommentTok}[1]{\textcolor[rgb]{0.37,0.37,0.37}{#1}}
\newcommand{\CommentVarTok}[1]{\textcolor[rgb]{0.37,0.37,0.37}{\textit{#1}}}
\newcommand{\ConstantTok}[1]{\textcolor[rgb]{0.56,0.35,0.01}{#1}}
\newcommand{\ControlFlowTok}[1]{\textcolor[rgb]{0.00,0.23,0.31}{\textbf{#1}}}
\newcommand{\DataTypeTok}[1]{\textcolor[rgb]{0.68,0.00,0.00}{#1}}
\newcommand{\DecValTok}[1]{\textcolor[rgb]{0.68,0.00,0.00}{#1}}
\newcommand{\DocumentationTok}[1]{\textcolor[rgb]{0.37,0.37,0.37}{\textit{#1}}}
\newcommand{\ErrorTok}[1]{\textcolor[rgb]{0.68,0.00,0.00}{#1}}
\newcommand{\ExtensionTok}[1]{\textcolor[rgb]{0.00,0.23,0.31}{#1}}
\newcommand{\FloatTok}[1]{\textcolor[rgb]{0.68,0.00,0.00}{#1}}
\newcommand{\FunctionTok}[1]{\textcolor[rgb]{0.28,0.35,0.67}{#1}}
\newcommand{\ImportTok}[1]{\textcolor[rgb]{0.00,0.46,0.62}{#1}}
\newcommand{\InformationTok}[1]{\textcolor[rgb]{0.37,0.37,0.37}{#1}}
\newcommand{\KeywordTok}[1]{\textcolor[rgb]{0.00,0.23,0.31}{\textbf{#1}}}
\newcommand{\NormalTok}[1]{\textcolor[rgb]{0.00,0.23,0.31}{#1}}
\newcommand{\OperatorTok}[1]{\textcolor[rgb]{0.37,0.37,0.37}{#1}}
\newcommand{\OtherTok}[1]{\textcolor[rgb]{0.00,0.23,0.31}{#1}}
\newcommand{\PreprocessorTok}[1]{\textcolor[rgb]{0.68,0.00,0.00}{#1}}
\newcommand{\RegionMarkerTok}[1]{\textcolor[rgb]{0.00,0.23,0.31}{#1}}
\newcommand{\SpecialCharTok}[1]{\textcolor[rgb]{0.37,0.37,0.37}{#1}}
\newcommand{\SpecialStringTok}[1]{\textcolor[rgb]{0.13,0.47,0.30}{#1}}
\newcommand{\StringTok}[1]{\textcolor[rgb]{0.13,0.47,0.30}{#1}}
\newcommand{\VariableTok}[1]{\textcolor[rgb]{0.07,0.07,0.07}{#1}}
\newcommand{\VerbatimStringTok}[1]{\textcolor[rgb]{0.13,0.47,0.30}{#1}}
\newcommand{\WarningTok}[1]{\textcolor[rgb]{0.37,0.37,0.37}{\textit{#1}}}

\usepackage{longtable,booktabs,array}
\usepackage{calc} % for calculating minipage widths
% Correct order of tables after \paragraph or \subparagraph
\usepackage{etoolbox}
\makeatletter
\patchcmd\longtable{\par}{\if@noskipsec\mbox{}\fi\par}{}{}
\makeatother
% Allow footnotes in longtable head/foot
\IfFileExists{footnotehyper.sty}{\usepackage{footnotehyper}}{\usepackage{footnote}}
\makesavenoteenv{longtable}
\usepackage{graphicx}
\makeatletter
\newsavebox\pandoc@box
\newcommand*\pandocbounded[1]{% scales image to fit in text height/width
  \sbox\pandoc@box{#1}%
  \Gscale@div\@tempa{\textheight}{\dimexpr\ht\pandoc@box+\dp\pandoc@box\relax}%
  \Gscale@div\@tempb{\linewidth}{\wd\pandoc@box}%
  \ifdim\@tempb\p@<\@tempa\p@\let\@tempa\@tempb\fi% select the smaller of both
  \ifdim\@tempa\p@<\p@\scalebox{\@tempa}{\usebox\pandoc@box}%
  \else\usebox{\pandoc@box}%
  \fi%
}
% Set default figure placement to htbp
\def\fps@figure{htbp}
\makeatother



\ifLuaTeX
\usepackage[bidi=basic]{babel}
\else
\usepackage[bidi=default]{babel}
\fi
% get rid of language-specific shorthands (see #6817):
\let\LanguageShortHands\languageshorthands
\def\languageshorthands#1{}


\setlength{\emergencystretch}{3em} % prevent overfull lines

\providecommand{\tightlist}{%
  \setlength{\itemsep}{0pt}\setlength{\parskip}{0pt}}



 


\KOMAoption{captions}{tableheading}
\makeatletter
\@ifpackageloaded{tcolorbox}{}{\usepackage[skins,breakable]{tcolorbox}}
\@ifpackageloaded{fontawesome5}{}{\usepackage{fontawesome5}}
\definecolor{quarto-callout-color}{HTML}{909090}
\definecolor{quarto-callout-note-color}{HTML}{0758E5}
\definecolor{quarto-callout-important-color}{HTML}{CC1914}
\definecolor{quarto-callout-warning-color}{HTML}{EB9113}
\definecolor{quarto-callout-tip-color}{HTML}{00A047}
\definecolor{quarto-callout-caution-color}{HTML}{FC5300}
\definecolor{quarto-callout-color-frame}{HTML}{acacac}
\definecolor{quarto-callout-note-color-frame}{HTML}{4582ec}
\definecolor{quarto-callout-important-color-frame}{HTML}{d9534f}
\definecolor{quarto-callout-warning-color-frame}{HTML}{f0ad4e}
\definecolor{quarto-callout-tip-color-frame}{HTML}{02b875}
\definecolor{quarto-callout-caution-color-frame}{HTML}{fd7e14}
\makeatother
\makeatletter
\@ifpackageloaded{caption}{}{\usepackage{caption}}
\AtBeginDocument{%
\ifdefined\contentsname
  \renewcommand*\contentsname{Tabla de contenidos}
\else
  \newcommand\contentsname{Tabla de contenidos}
\fi
\ifdefined\listfigurename
  \renewcommand*\listfigurename{Listado de Figuras}
\else
  \newcommand\listfigurename{Listado de Figuras}
\fi
\ifdefined\listtablename
  \renewcommand*\listtablename{Listado de Tablas}
\else
  \newcommand\listtablename{Listado de Tablas}
\fi
\ifdefined\figurename
  \renewcommand*\figurename{Figura}
\else
  \newcommand\figurename{Figura}
\fi
\ifdefined\tablename
  \renewcommand*\tablename{Tabla}
\else
  \newcommand\tablename{Tabla}
\fi
}
\@ifpackageloaded{float}{}{\usepackage{float}}
\floatstyle{ruled}
\@ifundefined{c@chapter}{\newfloat{codelisting}{h}{lop}}{\newfloat{codelisting}{h}{lop}[chapter]}
\floatname{codelisting}{Listado}
\newcommand*\listoflistings{\listof{codelisting}{Listado de Listados}}
\makeatother
\makeatletter
\makeatother
\makeatletter
\@ifpackageloaded{caption}{}{\usepackage{caption}}
\@ifpackageloaded{subcaption}{}{\usepackage{subcaption}}
\makeatother
\usepackage{bookmark}
\IfFileExists{xurl.sty}{\usepackage{xurl}}{} % add URL line breaks if available
\urlstyle{same}
\hypersetup{
  pdftitle={Capitulo\_3\_Regresion\_Multiple},
  pdfauthor={Econometría para la Gestión (ECO\_EPG) - FEN UAH},
  pdflang={es},
  colorlinks=true,
  linkcolor={blue},
  filecolor={Maroon},
  citecolor={Blue},
  urlcolor={Blue},
  pdfcreator={LaTeX via pandoc}}


\title{Capitulo\_3\_Regresion\_Multiple}
\author{Econometría para la Gestión (ECO\_EPG) - FEN UAH}
\date{}
\begin{document}
\maketitle

\renewcommand*\contentsname{Tabla de contenidos}
{
\hypersetup{linkcolor=}
\setcounter{tocdepth}{3}
\tableofcontents
}

\section{1. Material descargable}\label{material-descargable}

\href{pdf_epg/Capitulo_3_Regresion_Multiple.pdf}{Descargar PDF de
contenidos teóricos}

El documento incluye:

\begin{itemize}
\tightlist
\item
  Matriz de correlación y multicolinealidad.\\
\item
  Supuestos del modelo clásico de regresión múltiple.\\
\item
  Derivación de estimadores OLS ((X'X)\^{}\{-1\}X'Y).\\
\item
  Varianza del error y matriz de covarianza del estimador.\\
\item
  (R\^{}2), (R\^{}2) ajustado y prueba F global.\\
\item
  Inferencia individual sobre coeficientes ((t)-tests).\\
\item
  Intervalos de confianza.\\
\item
  Predicción:

  \begin{itemize}
  \tightlist
  \item
    Fórmula teórica ( \hat{y}\_0 = x\_0'\hat{\beta} )\\
  \item
    Intervalo: ( \hat{y}\_0 \pm s\sqrt{1+h_0}t ).\\
  \end{itemize}
\item
  Pruebas de autocorrelación y heterocedasticidad.
\end{itemize}

\section{Configuración inicial}\label{configuraciuxf3n-inicial}

\begin{Shaded}
\begin{Highlighting}[]
\FunctionTok{library}\NormalTok{(openxlsx)}
\FunctionTok{library}\NormalTok{(corrplot)}
\FunctionTok{library}\NormalTok{(lmtest)}
\end{Highlighting}
\end{Shaded}

Definimos tu ruta de datos:

\begin{Shaded}
\begin{Highlighting}[]
\NormalTok{ruta\_datos }\OtherTok{\textless{}{-}} \StringTok{"C:/Users/manue/Desktop/lab{-}econometria/labs\_epg/data\_epg"}
\FunctionTok{list.files}\NormalTok{(ruta\_datos)}
\end{Highlighting}
\end{Shaded}

\begin{verbatim}
 [1] "annos_mantenimiento.xlsx" "auto_peso_consumo.xlsx"  
 [3] "costos.xlsx"              "data_PCA_Decathlon.csv"  
 [5] "data_PCA_ExpertWine.csv"  "Ejemplo1.xlsx"           
 [7] "Ejemplo2.xlsx"            "millaje.txt"             
 [9] "orange.csv"               "tabla_ejemplo_R.xlsx"    
\end{verbatim}

\section{Lectura de datos}\label{lectura-de-datos}

\begin{Shaded}
\begin{Highlighting}[]
\NormalTok{archivo }\OtherTok{\textless{}{-}} \FunctionTok{file.path}\NormalTok{(ruta\_datos, }\StringTok{"costos.xlsx"}\NormalTok{)}
\NormalTok{datos }\OtherTok{\textless{}{-}} \FunctionTok{read.xlsx}\NormalTok{(archivo, }\AttributeTok{sheet=}\StringTok{"Hoja1"}\NormalTok{, }\AttributeTok{colNames=}\ConstantTok{TRUE}\NormalTok{)}
\FunctionTok{head}\NormalTok{(datos)}
\end{Highlighting}
\end{Shaded}

\begin{verbatim}
  Costos_generales Horas_mano_de_obra Horas_maquina Numero_preparaciones
1           155000                985          1060                  200
2           160000               1068          1080                  225
3           170000               1095          1100                  250
4           165000               1105          1200                  202
5           185000               1200          1600                  210
6           135000               1160          1100                  150
\end{verbatim}

Los datos contienen:

\begin{itemize}
\tightlist
\item
  \textbf{Costos\_generales} (variable dependiente).\\
\item
  \textbf{Horas\_maquina} y \textbf{Numero\_preparaciones} (variables
  explicativas).
\end{itemize}

\section{Matriz de correlación}\label{matriz-de-correlaciuxf3n}

La matriz de correlación nos permite:

\begin{itemize}
\tightlist
\item
  Detectar \textbf{relaciones lineales} entre pares de variables.\\
\item
  Identificar \textbf{multicolinealidad} entre regresores.
\end{itemize}

\begin{Shaded}
\begin{Highlighting}[]
\NormalTok{r }\OtherTok{\textless{}{-}} \FunctionTok{cor}\NormalTok{(datos)}
\NormalTok{r}
\end{Highlighting}
\end{Shaded}

\begin{verbatim}
                     Costos_generales Horas_mano_de_obra Horas_maquina
Costos_generales            1.0000000         0.53462157     0.8775005
Horas_mano_de_obra          0.5346216         1.00000000     0.7680722
Horas_maquina               0.8775005         0.76807222     1.0000000
Numero_preparaciones        0.6274762        -0.04731148     0.2507478
                     Numero_preparaciones
Costos_generales               0.62747618
Horas_mano_de_obra            -0.04731148
Horas_maquina                  0.25074784
Numero_preparaciones           1.00000000
\end{verbatim}

\subsubsection{Gráficos de
exploración}\label{gruxe1ficos-de-exploraciuxf3n}

\begin{Shaded}
\begin{Highlighting}[]
\FunctionTok{pairs}\NormalTok{(datos)}
\end{Highlighting}
\end{Shaded}

\begin{center}
\pandocbounded{\includegraphics[keepaspectratio]{lab03_epg_files/figure-pdf/unnamed-chunk-5-1.pdf}}
\end{center}

\begin{Shaded}
\begin{Highlighting}[]
\FunctionTok{corrplot}\NormalTok{(r, }\AttributeTok{method=}\StringTok{"circle"}\NormalTok{, }\AttributeTok{type=}\StringTok{"lower"}\NormalTok{, }\AttributeTok{diag=}\ConstantTok{FALSE}\NormalTok{,}
         \AttributeTok{tl.col=}\StringTok{"black"}\NormalTok{, }\AttributeTok{tl.cex=}\DecValTok{1}\NormalTok{, }\AttributeTok{tl.srt=}\DecValTok{45}\NormalTok{)}
\end{Highlighting}
\end{Shaded}

\begin{center}
\pandocbounded{\includegraphics[keepaspectratio]{lab03_epg_files/figure-pdf/unnamed-chunk-5-2.pdf}}
\end{center}

\begin{tcolorbox}[enhanced jigsaw, opacitybacktitle=0.6, toprule=.15mm, colbacktitle=quarto-callout-note-color!10!white, coltitle=black, colframe=quarto-callout-note-color-frame, opacityback=0, breakable, toptitle=1mm, arc=.35mm, left=2mm, leftrule=.75mm, titlerule=0mm, bottomtitle=1mm, title=\textcolor{quarto-callout-note-color}{\faInfo}\hspace{0.5em}{Nota}, colback=white, rightrule=.15mm, bottomrule=.15mm]

🔍 \textbf{Interpretación pedagógica:}\\
- Valores altos en la correlación entre regresores →
\textbf{multicolinealidad}.\\
- Esto puede inflar varianzas de los estimadores y hacer inestables los
coeficientes.\\
- La matriz y el \texttt{corrplot} permiten detectar problemas antes de
ajustar el modelo.

\end{tcolorbox}

\section{Modelo de regresión
múltiple}\label{modelo-de-regresiuxf3n-muxfaltiple}

Ajustamos el modelo:

\[
\text{Costos} = \beta_0 + \beta_1(\text{Horas\_maquina}) + \beta_2(\text{Numero\_preparaciones}) + \varepsilon
\]

\begin{Shaded}
\begin{Highlighting}[]
\NormalTok{modelo }\OtherTok{\textless{}{-}} \FunctionTok{lm}\NormalTok{(Costos\_generales }\SpecialCharTok{\textasciitilde{}}\NormalTok{ Horas\_maquina }\SpecialCharTok{+}\NormalTok{ Numero\_preparaciones, }\AttributeTok{data=}\NormalTok{datos)}
\FunctionTok{summary}\NormalTok{(modelo)}
\end{Highlighting}
\end{Shaded}

\begin{verbatim}

Call:
lm(formula = Costos_generales ~ Horas_maquina + Numero_preparaciones, 
    data = datos)

Residuals:
   Min     1Q Median     3Q    Max 
 -7157  -2827    768   1449   9407 

Coefficients:
                     Estimate Std. Error t value Pr(>|t|)    
(Intercept)          19796.44   12787.83   1.548 0.156013    
Horas_maquina           65.44       6.74   9.709 4.57e-06 ***
Numero_preparaciones   322.21      58.66   5.493 0.000384 ***
---
Signif. codes:  0 '***' 0.001 '**' 0.01 '*' 0.05 '.' 0.1 ' ' 1

Residual standard error: 4951 on 9 degrees of freedom
Multiple R-squared:  0.9472,    Adjusted R-squared:  0.9354 
F-statistic: 80.66 on 2 and 9 DF,  p-value: 1.792e-06
\end{verbatim}

\begin{tcolorbox}[enhanced jigsaw, opacitybacktitle=0.6, toprule=.15mm, colbacktitle=quarto-callout-note-color!10!white, coltitle=black, colframe=quarto-callout-note-color-frame, opacityback=0, breakable, toptitle=1mm, arc=.35mm, left=2mm, leftrule=.75mm, titlerule=0mm, bottomtitle=1mm, title=\textcolor{quarto-callout-note-color}{\faInfo}\hspace{0.5em}{Nota}, colback=white, rightrule=.15mm, bottomrule=.15mm]

\begin{itemize}
\tightlist
\item
  \textbf{Prueba individual (t test):} evalúa si cada (\beta\_j) es
  distinto de 0.\\
\item
  \textbf{Prueba global (F test):} evalúa si el modelo en conjunto
  explica la variable dependiente.\\
\item
  **Signo de (\beta\_j):** indica dirección del efecto.\\
\item
  \textbf{Magnitud:} indica cuánto cambia Y ante un cambio unitario en
  el regresor.\\
\item
  \textbf{p-values pequeños:} evidencia estadística de relación
  significativa.
\end{itemize}

\end{tcolorbox}

\section{Residuos del modelo}\label{residuos-del-modelo}

\begin{Shaded}
\begin{Highlighting}[]
\NormalTok{epsilon }\OtherTok{\textless{}{-}}\NormalTok{ modelo}\SpecialCharTok{$}\NormalTok{residuals}
\FunctionTok{hist}\NormalTok{(epsilon)}
\end{Highlighting}
\end{Shaded}

\begin{center}
\pandocbounded{\includegraphics[keepaspectratio]{lab03_epg_files/figure-pdf/unnamed-chunk-7-1.pdf}}
\end{center}

\begin{Shaded}
\begin{Highlighting}[]
\FunctionTok{plot}\NormalTok{(}\FunctionTok{density}\NormalTok{(epsilon))}
\end{Highlighting}
\end{Shaded}

\begin{center}
\pandocbounded{\includegraphics[keepaspectratio]{lab03_epg_files/figure-pdf/unnamed-chunk-7-2.pdf}}
\end{center}

\begin{Shaded}
\begin{Highlighting}[]
\FunctionTok{mean}\NormalTok{(epsilon)}
\end{Highlighting}
\end{Shaded}

\begin{verbatim}
[1] 1.705303e-13
\end{verbatim}

\subsubsection{Normalidad de los
residuos}\label{normalidad-de-los-residuos}

\begin{Shaded}
\begin{Highlighting}[]
\FunctionTok{shapiro.test}\NormalTok{(epsilon)}
\end{Highlighting}
\end{Shaded}

\begin{verbatim}

    Shapiro-Wilk normality test

data:  epsilon
W = 0.95577, p-value = 0.7222
\end{verbatim}

\subsubsection{Autocorrelación
(Durbin--Watson)}\label{autocorrelaciuxf3n-durbinwatson}

\begin{Shaded}
\begin{Highlighting}[]
\FunctionTok{dwtest}\NormalTok{(modelo, }\AttributeTok{alternative=}\StringTok{"two.sided"}\NormalTok{, }\AttributeTok{iterations=}\DecValTok{1000}\NormalTok{)}
\end{Highlighting}
\end{Shaded}

\begin{verbatim}

    Durbin-Watson test

data:  modelo
DW = 2.0815, p-value = 0.7416
alternative hypothesis: true autocorrelation is not 0
\end{verbatim}

Interpretación: - DW ≈ 2 → no autocorrelación\\
- DW \textless{} 2 → autocorrelación positiva\\
- DW \textgreater{} 2 → autocorrelación negativa

\subsubsection{Heterocedasticidad
(Breusch--Pagan)}\label{heterocedasticidad-breuschpagan}

\begin{Shaded}
\begin{Highlighting}[]
\FunctionTok{bptest}\NormalTok{(modelo)}
\end{Highlighting}
\end{Shaded}

\begin{verbatim}

    studentized Breusch-Pagan test

data:  modelo
BP = 1.0153, df = 2, p-value = 0.6019
\end{verbatim}

\section{Estadístico F calculado a
mano}\label{estaduxedstico-f-calculado-a-mano}

\begin{Shaded}
\begin{Highlighting}[]
\NormalTok{R2 }\OtherTok{\textless{}{-}} \FloatTok{0.9472}
\NormalTok{F }\OtherTok{\textless{}{-}}\NormalTok{ (R2}\SpecialCharTok{/}\DecValTok{2}\NormalTok{)}\SpecialCharTok{/}\NormalTok{((}\DecValTok{1}\SpecialCharTok{{-}}\NormalTok{R2)}\SpecialCharTok{/}\NormalTok{(}\DecValTok{12{-}3}\NormalTok{))}
\NormalTok{F}
\end{Highlighting}
\end{Shaded}

\begin{verbatim}
[1] 80.72727
\end{verbatim}

\begin{Shaded}
\begin{Highlighting}[]
\FunctionTok{qf}\NormalTok{(}\FloatTok{0.05}\NormalTok{, }\DecValTok{2}\NormalTok{, }\DecValTok{9}\NormalTok{, }\AttributeTok{lower.tail=}\ConstantTok{FALSE}\NormalTok{)}
\end{Highlighting}
\end{Shaded}

\begin{verbatim}
[1] 4.256495
\end{verbatim}

Interpretación: si F calculado \textgreater{} F crítico → el modelo
\textbf{aporta información significativa}.

\section{Desviación estándar del
error}\label{desviaciuxf3n-estuxe1ndar-del-error}

\begin{Shaded}
\begin{Highlighting}[]
\NormalTok{s }\OtherTok{\textless{}{-}} \FunctionTok{sqrt}\NormalTok{(}\FunctionTok{sum}\NormalTok{(epsilon}\SpecialCharTok{\^{}}\DecValTok{2}\NormalTok{)}\SpecialCharTok{/}\NormalTok{(}\DecValTok{12{-}2{-}1}\NormalTok{))}
\NormalTok{s}
\end{Highlighting}
\end{Shaded}

\begin{verbatim}
[1] 4951.106
\end{verbatim}

\section{Intervalos de confianza de los
coeficientes}\label{intervalos-de-confianza-de-los-coeficientes}

\begin{Shaded}
\begin{Highlighting}[]
\FunctionTok{confint}\NormalTok{(modelo)}
\end{Highlighting}
\end{Shaded}

\begin{verbatim}
                           2.5 %     97.5 %
(Intercept)          -9131.64046 48724.5095
Horas_maquina           50.18894    80.6827
Numero_preparaciones   189.50994   454.9046
\end{verbatim}

\subsubsection{Cálculo manual del
intervalo}\label{cuxe1lculo-manual-del-intervalo}

\begin{Shaded}
\begin{Highlighting}[]
\NormalTok{se }\OtherTok{\textless{}{-}} \FunctionTok{sqrt}\NormalTok{(}\FunctionTok{diag}\NormalTok{(}\FunctionTok{vcov}\NormalTok{(modelo)))}
\NormalTok{t }\OtherTok{\textless{}{-}} \SpecialCharTok{{-}}\DecValTok{1}\SpecialCharTok{*}\FunctionTok{qt}\NormalTok{(}\FloatTok{0.025}\NormalTok{, }\DecValTok{12{-}2{-}1}\NormalTok{, }\AttributeTok{lower.tail=}\ConstantTok{FALSE}\NormalTok{)}

\NormalTok{BHM }\OtherTok{\textless{}{-}} \FloatTok{65.44}
\NormalTok{sbhm }\OtherTok{\textless{}{-}} \FloatTok{6.74}

\NormalTok{limitsupBHM }\OtherTok{\textless{}{-}}\NormalTok{ BHM }\SpecialCharTok{+}\NormalTok{ t}\SpecialCharTok{*}\NormalTok{sbhm}
\NormalTok{limitinfBHM }\OtherTok{\textless{}{-}}\NormalTok{ BHM }\SpecialCharTok{{-}}\NormalTok{ t}\SpecialCharTok{*}\NormalTok{sbhm}
\NormalTok{limitsupBHM; limitinfBHM}
\end{Highlighting}
\end{Shaded}

\begin{verbatim}
[1] 50.19306
\end{verbatim}

\begin{verbatim}
[1] 80.68694
\end{verbatim}

\section{Intervalo para la varianza del
error}\label{intervalo-para-la-varianza-del-error}

\begin{Shaded}
\begin{Highlighting}[]
\NormalTok{gamma1 }\OtherTok{\textless{}{-}} \FunctionTok{qchisq}\NormalTok{(}\FloatTok{0.025}\NormalTok{, }\DecValTok{12{-}2{-}1}\NormalTok{)}
\NormalTok{gamma2 }\OtherTok{\textless{}{-}} \FunctionTok{qchisq}\NormalTok{(}\FloatTok{0.975}\NormalTok{, }\DecValTok{12{-}2{-}1}\NormalTok{)}

\NormalTok{s2\_LI }\OtherTok{\textless{}{-}}\NormalTok{ (}\DecValTok{12{-}2{-}1}\NormalTok{) }\SpecialCharTok{*}\NormalTok{ s}\SpecialCharTok{\^{}}\DecValTok{2} \SpecialCharTok{/}\NormalTok{ gamma2}
\NormalTok{s2\_LS }\OtherTok{\textless{}{-}}\NormalTok{ (}\DecValTok{12{-}2{-}1}\NormalTok{) }\SpecialCharTok{*}\NormalTok{ s}\SpecialCharTok{\^{}}\DecValTok{2} \SpecialCharTok{/}\NormalTok{ gamma1}

\NormalTok{s2\_LI; s2\_LS}
\end{Highlighting}
\end{Shaded}

\begin{verbatim}
[1] 11597739
\end{verbatim}

\begin{verbatim}
[1] 81699732
\end{verbatim}

\section{Predicción puntual}\label{predicciuxf3n-puntual}

\begin{Shaded}
\begin{Highlighting}[]
\NormalTok{nuevo }\OtherTok{\textless{}{-}} \FunctionTok{data.frame}\NormalTok{(}\AttributeTok{Horas\_maquina=}\DecValTok{2000}\NormalTok{, }\AttributeTok{Numero\_preparaciones=}\DecValTok{220}\NormalTok{)}

\NormalTok{valor\_predicho }\OtherTok{\textless{}{-}} \FunctionTok{predict}\NormalTok{(modelo, }\AttributeTok{newdata=}\NormalTok{nuevo)}
\NormalTok{valor\_predicho}
\end{Highlighting}
\end{Shaded}

\begin{verbatim}
       1 
221553.7 
\end{verbatim}

\section{\texorpdfstring{Intervalo de confianza con
\texttt{predict}}{Intervalo de confianza con predict}}\label{intervalo-de-confianza-con-predict}

\begin{Shaded}
\begin{Highlighting}[]
\NormalTok{valor\_predicho2 }\OtherTok{\textless{}{-}} \FunctionTok{predict}\NormalTok{(modelo, }\AttributeTok{newdata=}\NormalTok{nuevo, }\AttributeTok{interval=}\StringTok{"confidence"}\NormalTok{)}
\NormalTok{valor\_predicho2}
\end{Highlighting}
\end{Shaded}

\begin{verbatim}
       fit      lwr      upr
1 221553.7 210265.6 232841.8
\end{verbatim}

\section{Predicción manual paso a
paso}\label{predicciuxf3n-manual-paso-a-paso}

Construimos la matriz (X):

\begin{Shaded}
\begin{Highlighting}[]
\NormalTok{X }\OtherTok{\textless{}{-}} \FunctionTok{cbind}\NormalTok{(}\DecValTok{1}\NormalTok{, datos}\SpecialCharTok{$}\NormalTok{Horas\_maquina, datos}\SpecialCharTok{$}\NormalTok{Numero\_preparaciones)}
\NormalTok{M }\OtherTok{\textless{}{-}} \FunctionTok{solve}\NormalTok{(}\FunctionTok{t}\NormalTok{(X) }\SpecialCharTok{\%*\%}\NormalTok{ X)}
\NormalTok{beta }\OtherTok{\textless{}{-}}\NormalTok{ M }\SpecialCharTok{\%*\%} \FunctionTok{t}\NormalTok{(X) }\SpecialCharTok{\%*\%}\NormalTok{ datos}\SpecialCharTok{$}\NormalTok{Costos\_generales}

\NormalTok{x0 }\OtherTok{\textless{}{-}} \FunctionTok{c}\NormalTok{(}\DecValTok{1}\NormalTok{, }\DecValTok{2000}\NormalTok{, }\DecValTok{220}\NormalTok{)}
\NormalTok{h0 }\OtherTok{\textless{}{-}} \FunctionTok{t}\NormalTok{(x0) }\SpecialCharTok{\%*\%}\NormalTok{ M }\SpecialCharTok{\%*\%}\NormalTok{ x0}

\NormalTok{y0 }\OtherTok{\textless{}{-}} \FunctionTok{t}\NormalTok{(beta) }\SpecialCharTok{\%*\%}\NormalTok{ x0}
\NormalTok{y0}
\end{Highlighting}
\end{Shaded}

\begin{verbatim}
         [,1]
[1,] 221553.7
\end{verbatim}

\subsubsection{Intervalo manual}\label{intervalo-manual}

\begin{Shaded}
\begin{Highlighting}[]
\NormalTok{y\_limsup }\OtherTok{\textless{}{-}}\NormalTok{ y0 }\SpecialCharTok{+}\NormalTok{ s}\SpecialCharTok{*}\FunctionTok{sqrt}\NormalTok{(}\DecValTok{1}\SpecialCharTok{+}\NormalTok{h0)}\SpecialCharTok{*}\FunctionTok{qt}\NormalTok{(}\FloatTok{0.975}\NormalTok{, }\DecValTok{12{-}2{-}1}\NormalTok{, }\AttributeTok{lower.tail=}\ConstantTok{FALSE}\NormalTok{)}
\NormalTok{y\_liminf }\OtherTok{\textless{}{-}}\NormalTok{ y0 }\SpecialCharTok{{-}}\NormalTok{ s}\SpecialCharTok{*}\FunctionTok{sqrt}\NormalTok{(}\DecValTok{1}\SpecialCharTok{+}\NormalTok{h0)}\SpecialCharTok{*}\FunctionTok{qt}\NormalTok{(}\FloatTok{0.975}\NormalTok{, }\DecValTok{12{-}2{-}1}\NormalTok{, }\AttributeTok{lower.tail=}\ConstantTok{FALSE}\NormalTok{)}

\NormalTok{y\_liminf; y\_limsup}
\end{Highlighting}
\end{Shaded}

\begin{verbatim}
         [,1]
[1,] 237455.4
\end{verbatim}

\begin{verbatim}
         [,1]
[1,] 205651.9
\end{verbatim}

\section{Gráfico Y real vs Y
predicho}\label{gruxe1fico-y-real-vs-y-predicho}

\begin{Shaded}
\begin{Highlighting}[]
\FunctionTok{plot}\NormalTok{(modelo}\SpecialCharTok{$}\NormalTok{fitted.values, datos}\SpecialCharTok{$}\NormalTok{Costos\_generales,}
     \AttributeTok{main=}\StringTok{"Revisión Valor Real vs Valor Predicho"}\NormalTok{)}
\FunctionTok{lines}\NormalTok{(}\FunctionTok{c}\NormalTok{(}\DecValTok{140000}\NormalTok{,}\DecValTok{200000}\NormalTok{), }\FunctionTok{c}\NormalTok{(}\DecValTok{140000}\NormalTok{,}\DecValTok{200000}\NormalTok{))}
\end{Highlighting}
\end{Shaded}

\begin{center}
\pandocbounded{\includegraphics[keepaspectratio]{lab03_epg_files/figure-pdf/unnamed-chunk-20-1.pdf}}
\end{center}

\section{Modelo sin constante}\label{modelo-sin-constante}

\begin{Shaded}
\begin{Highlighting}[]
\NormalTok{modelo2 }\OtherTok{\textless{}{-}} \FunctionTok{lm}\NormalTok{(Costos\_generales }\SpecialCharTok{\textasciitilde{}}\NormalTok{ Horas\_maquina }\SpecialCharTok{+}\NormalTok{ Numero\_preparaciones }\SpecialCharTok{{-}} \DecValTok{1}\NormalTok{, }\AttributeTok{data=}\NormalTok{datos)}
\FunctionTok{summary}\NormalTok{(modelo2)}
\end{Highlighting}
\end{Shaded}

\begin{verbatim}

Call:
lm(formula = Costos_generales ~ Horas_maquina + Numero_preparaciones - 
    1, data = datos)

Residuals:
    Min      1Q  Median      3Q     Max 
-9042.2 -3486.9   739.7  3467.0  9300.3 

Coefficients:
                     Estimate Std. Error t value Pr(>|t|)    
Horas_maquina          69.994      6.472  10.814 7.72e-07 ***
Numero_preparaciones  390.723     41.098   9.507 2.52e-06 ***
---
Signif. codes:  0 '***' 0.001 '**' 0.01 '*' 0.05 '.' 0.1 ' ' 1

Residual standard error: 5286 on 10 degrees of freedom
Multiple R-squared:  0.9992,    Adjusted R-squared:  0.999 
F-statistic:  6095 on 2 and 10 DF,  p-value: 3.699e-16
\end{verbatim}

\subsubsection{Análisis de residuos modelo sin
constante}\label{anuxe1lisis-de-residuos-modelo-sin-constante}

\begin{Shaded}
\begin{Highlighting}[]
\NormalTok{epsilon }\OtherTok{\textless{}{-}}\NormalTok{ modelo2}\SpecialCharTok{$}\NormalTok{residuals}
\FunctionTok{hist}\NormalTok{(epsilon)}
\end{Highlighting}
\end{Shaded}

\begin{center}
\pandocbounded{\includegraphics[keepaspectratio]{lab03_epg_files/figure-pdf/unnamed-chunk-22-1.pdf}}
\end{center}

\begin{Shaded}
\begin{Highlighting}[]
\FunctionTok{plot}\NormalTok{(}\FunctionTok{density}\NormalTok{(epsilon))}
\end{Highlighting}
\end{Shaded}

\begin{center}
\pandocbounded{\includegraphics[keepaspectratio]{lab03_epg_files/figure-pdf/unnamed-chunk-22-2.pdf}}
\end{center}

\begin{Shaded}
\begin{Highlighting}[]
\FunctionTok{shapiro.test}\NormalTok{(epsilon)}
\end{Highlighting}
\end{Shaded}

\begin{verbatim}

    Shapiro-Wilk normality test

data:  epsilon
W = 0.97973, p-value = 0.9825
\end{verbatim}

\begin{Shaded}
\begin{Highlighting}[]
\FunctionTok{dwtest}\NormalTok{(modelo2,}\AttributeTok{alternative=}\StringTok{"two.sided"}\NormalTok{,}\AttributeTok{iterations=}\DecValTok{1000}\NormalTok{)}
\end{Highlighting}
\end{Shaded}

\begin{verbatim}

    Durbin-Watson test

data:  modelo2
DW = 2.1284, p-value = 0.9776
alternative hypothesis: true autocorrelation is not 0
\end{verbatim}

\begin{Shaded}
\begin{Highlighting}[]
\FunctionTok{bptest}\NormalTok{(modelo2)}
\end{Highlighting}
\end{Shaded}

\begin{verbatim}

    studentized Breusch-Pagan test

data:  modelo2
BP = 0.33027, df = 1, p-value = 0.5655
\end{verbatim}




\end{document}
