% Options for packages loaded elsewhere
% Options for packages loaded elsewhere
\PassOptionsToPackage{unicode}{hyperref}
\PassOptionsToPackage{hyphens}{url}
\PassOptionsToPackage{dvipsnames,svgnames,x11names}{xcolor}
%
\documentclass[
  spanish,
  letterpaper,
  DIV=11,
  numbers=noendperiod]{scrartcl}
\usepackage{xcolor}
\usepackage{amsmath,amssymb}
\setcounter{secnumdepth}{5}
\usepackage{iftex}
\ifPDFTeX
  \usepackage[T1]{fontenc}
  \usepackage[utf8]{inputenc}
  \usepackage{textcomp} % provide euro and other symbols
\else % if luatex or xetex
  \usepackage{unicode-math} % this also loads fontspec
  \defaultfontfeatures{Scale=MatchLowercase}
  \defaultfontfeatures[\rmfamily]{Ligatures=TeX,Scale=1}
\fi
\usepackage{lmodern}
\ifPDFTeX\else
  % xetex/luatex font selection
\fi
% Use upquote if available, for straight quotes in verbatim environments
\IfFileExists{upquote.sty}{\usepackage{upquote}}{}
\IfFileExists{microtype.sty}{% use microtype if available
  \usepackage[]{microtype}
  \UseMicrotypeSet[protrusion]{basicmath} % disable protrusion for tt fonts
}{}
\makeatletter
\@ifundefined{KOMAClassName}{% if non-KOMA class
  \IfFileExists{parskip.sty}{%
    \usepackage{parskip}
  }{% else
    \setlength{\parindent}{0pt}
    \setlength{\parskip}{6pt plus 2pt minus 1pt}}
}{% if KOMA class
  \KOMAoptions{parskip=half}}
\makeatother
% Make \paragraph and \subparagraph free-standing
\makeatletter
\ifx\paragraph\undefined\else
  \let\oldparagraph\paragraph
  \renewcommand{\paragraph}{
    \@ifstar
      \xxxParagraphStar
      \xxxParagraphNoStar
  }
  \newcommand{\xxxParagraphStar}[1]{\oldparagraph*{#1}\mbox{}}
  \newcommand{\xxxParagraphNoStar}[1]{\oldparagraph{#1}\mbox{}}
\fi
\ifx\subparagraph\undefined\else
  \let\oldsubparagraph\subparagraph
  \renewcommand{\subparagraph}{
    \@ifstar
      \xxxSubParagraphStar
      \xxxSubParagraphNoStar
  }
  \newcommand{\xxxSubParagraphStar}[1]{\oldsubparagraph*{#1}\mbox{}}
  \newcommand{\xxxSubParagraphNoStar}[1]{\oldsubparagraph{#1}\mbox{}}
\fi
\makeatother

\usepackage{color}
\usepackage{fancyvrb}
\newcommand{\VerbBar}{|}
\newcommand{\VERB}{\Verb[commandchars=\\\{\}]}
\DefineVerbatimEnvironment{Highlighting}{Verbatim}{commandchars=\\\{\}}
% Add ',fontsize=\small' for more characters per line
\usepackage{framed}
\definecolor{shadecolor}{RGB}{241,243,245}
\newenvironment{Shaded}{\begin{snugshade}}{\end{snugshade}}
\newcommand{\AlertTok}[1]{\textcolor[rgb]{0.68,0.00,0.00}{#1}}
\newcommand{\AnnotationTok}[1]{\textcolor[rgb]{0.37,0.37,0.37}{#1}}
\newcommand{\AttributeTok}[1]{\textcolor[rgb]{0.40,0.45,0.13}{#1}}
\newcommand{\BaseNTok}[1]{\textcolor[rgb]{0.68,0.00,0.00}{#1}}
\newcommand{\BuiltInTok}[1]{\textcolor[rgb]{0.00,0.23,0.31}{#1}}
\newcommand{\CharTok}[1]{\textcolor[rgb]{0.13,0.47,0.30}{#1}}
\newcommand{\CommentTok}[1]{\textcolor[rgb]{0.37,0.37,0.37}{#1}}
\newcommand{\CommentVarTok}[1]{\textcolor[rgb]{0.37,0.37,0.37}{\textit{#1}}}
\newcommand{\ConstantTok}[1]{\textcolor[rgb]{0.56,0.35,0.01}{#1}}
\newcommand{\ControlFlowTok}[1]{\textcolor[rgb]{0.00,0.23,0.31}{\textbf{#1}}}
\newcommand{\DataTypeTok}[1]{\textcolor[rgb]{0.68,0.00,0.00}{#1}}
\newcommand{\DecValTok}[1]{\textcolor[rgb]{0.68,0.00,0.00}{#1}}
\newcommand{\DocumentationTok}[1]{\textcolor[rgb]{0.37,0.37,0.37}{\textit{#1}}}
\newcommand{\ErrorTok}[1]{\textcolor[rgb]{0.68,0.00,0.00}{#1}}
\newcommand{\ExtensionTok}[1]{\textcolor[rgb]{0.00,0.23,0.31}{#1}}
\newcommand{\FloatTok}[1]{\textcolor[rgb]{0.68,0.00,0.00}{#1}}
\newcommand{\FunctionTok}[1]{\textcolor[rgb]{0.28,0.35,0.67}{#1}}
\newcommand{\ImportTok}[1]{\textcolor[rgb]{0.00,0.46,0.62}{#1}}
\newcommand{\InformationTok}[1]{\textcolor[rgb]{0.37,0.37,0.37}{#1}}
\newcommand{\KeywordTok}[1]{\textcolor[rgb]{0.00,0.23,0.31}{\textbf{#1}}}
\newcommand{\NormalTok}[1]{\textcolor[rgb]{0.00,0.23,0.31}{#1}}
\newcommand{\OperatorTok}[1]{\textcolor[rgb]{0.37,0.37,0.37}{#1}}
\newcommand{\OtherTok}[1]{\textcolor[rgb]{0.00,0.23,0.31}{#1}}
\newcommand{\PreprocessorTok}[1]{\textcolor[rgb]{0.68,0.00,0.00}{#1}}
\newcommand{\RegionMarkerTok}[1]{\textcolor[rgb]{0.00,0.23,0.31}{#1}}
\newcommand{\SpecialCharTok}[1]{\textcolor[rgb]{0.37,0.37,0.37}{#1}}
\newcommand{\SpecialStringTok}[1]{\textcolor[rgb]{0.13,0.47,0.30}{#1}}
\newcommand{\StringTok}[1]{\textcolor[rgb]{0.13,0.47,0.30}{#1}}
\newcommand{\VariableTok}[1]{\textcolor[rgb]{0.07,0.07,0.07}{#1}}
\newcommand{\VerbatimStringTok}[1]{\textcolor[rgb]{0.13,0.47,0.30}{#1}}
\newcommand{\WarningTok}[1]{\textcolor[rgb]{0.37,0.37,0.37}{\textit{#1}}}

\usepackage{longtable,booktabs,array}
\usepackage{calc} % for calculating minipage widths
% Correct order of tables after \paragraph or \subparagraph
\usepackage{etoolbox}
\makeatletter
\patchcmd\longtable{\par}{\if@noskipsec\mbox{}\fi\par}{}{}
\makeatother
% Allow footnotes in longtable head/foot
\IfFileExists{footnotehyper.sty}{\usepackage{footnotehyper}}{\usepackage{footnote}}
\makesavenoteenv{longtable}
\usepackage{graphicx}
\makeatletter
\newsavebox\pandoc@box
\newcommand*\pandocbounded[1]{% scales image to fit in text height/width
  \sbox\pandoc@box{#1}%
  \Gscale@div\@tempa{\textheight}{\dimexpr\ht\pandoc@box+\dp\pandoc@box\relax}%
  \Gscale@div\@tempb{\linewidth}{\wd\pandoc@box}%
  \ifdim\@tempb\p@<\@tempa\p@\let\@tempa\@tempb\fi% select the smaller of both
  \ifdim\@tempa\p@<\p@\scalebox{\@tempa}{\usebox\pandoc@box}%
  \else\usebox{\pandoc@box}%
  \fi%
}
% Set default figure placement to htbp
\def\fps@figure{htbp}
\makeatother



\ifLuaTeX
\usepackage[bidi=basic]{babel}
\else
\usepackage[bidi=default]{babel}
\fi
% get rid of language-specific shorthands (see #6817):
\let\LanguageShortHands\languageshorthands
\def\languageshorthands#1{}


\setlength{\emergencystretch}{3em} % prevent overfull lines

\providecommand{\tightlist}{%
  \setlength{\itemsep}{0pt}\setlength{\parskip}{0pt}}



 


\KOMAoption{captions}{tableheading}
\makeatletter
\@ifpackageloaded{tcolorbox}{}{\usepackage[skins,breakable]{tcolorbox}}
\@ifpackageloaded{fontawesome5}{}{\usepackage{fontawesome5}}
\definecolor{quarto-callout-color}{HTML}{909090}
\definecolor{quarto-callout-note-color}{HTML}{0758E5}
\definecolor{quarto-callout-important-color}{HTML}{CC1914}
\definecolor{quarto-callout-warning-color}{HTML}{EB9113}
\definecolor{quarto-callout-tip-color}{HTML}{00A047}
\definecolor{quarto-callout-caution-color}{HTML}{FC5300}
\definecolor{quarto-callout-color-frame}{HTML}{acacac}
\definecolor{quarto-callout-note-color-frame}{HTML}{4582ec}
\definecolor{quarto-callout-important-color-frame}{HTML}{d9534f}
\definecolor{quarto-callout-warning-color-frame}{HTML}{f0ad4e}
\definecolor{quarto-callout-tip-color-frame}{HTML}{02b875}
\definecolor{quarto-callout-caution-color-frame}{HTML}{fd7e14}
\makeatother
\makeatletter
\@ifpackageloaded{caption}{}{\usepackage{caption}}
\AtBeginDocument{%
\ifdefined\contentsname
  \renewcommand*\contentsname{Tabla de contenidos}
\else
  \newcommand\contentsname{Tabla de contenidos}
\fi
\ifdefined\listfigurename
  \renewcommand*\listfigurename{Listado de Figuras}
\else
  \newcommand\listfigurename{Listado de Figuras}
\fi
\ifdefined\listtablename
  \renewcommand*\listtablename{Listado de Tablas}
\else
  \newcommand\listtablename{Listado de Tablas}
\fi
\ifdefined\figurename
  \renewcommand*\figurename{Figura}
\else
  \newcommand\figurename{Figura}
\fi
\ifdefined\tablename
  \renewcommand*\tablename{Tabla}
\else
  \newcommand\tablename{Tabla}
\fi
}
\@ifpackageloaded{float}{}{\usepackage{float}}
\floatstyle{ruled}
\@ifundefined{c@chapter}{\newfloat{codelisting}{h}{lop}}{\newfloat{codelisting}{h}{lop}[chapter]}
\floatname{codelisting}{Listado}
\newcommand*\listoflistings{\listof{codelisting}{Listado de Listados}}
\makeatother
\makeatletter
\makeatother
\makeatletter
\@ifpackageloaded{caption}{}{\usepackage{caption}}
\@ifpackageloaded{subcaption}{}{\usepackage{subcaption}}
\makeatother
\usepackage{bookmark}
\IfFileExists{xurl.sty}{\usepackage{xurl}}{} % add URL line breaks if available
\urlstyle{same}
\hypersetup{
  pdftitle={Capitulo 2 Correlación y Regresión Simple},
  pdfauthor={Econometría para la Gestión (ECO\_EPG) - FEN UAH},
  pdflang={es},
  colorlinks=true,
  linkcolor={blue},
  filecolor={Maroon},
  citecolor={Blue},
  urlcolor={Blue},
  pdfcreator={LaTeX via pandoc}}


\title{Capitulo 2 Correlación y Regresión Simple}
\author{Econometría para la Gestión (ECO\_EPG) - FEN UAH}
\date{}
\begin{document}
\maketitle

\renewcommand*\contentsname{Tabla de contenidos}
{
\hypersetup{linkcolor=}
\setcounter{tocdepth}{3}
\tableofcontents
}

\section{1. Material descargable}\label{material-descargable}

\href{pdf_epg/Capitulo_2_Correlacion_regresion_simple.pdf}{Descargar PDF
de contenidos teóricos}

El PDF \textbf{``Capitulo\_2\_Correlacion\_regresion\_simple''}
desarrolla los siguientes temas principales (a modo de índice):

\begin{itemize}
\tightlist
\item
  Covarianza y correlación.\\
\item
  Diagramas de dispersión.\\
\item
  Prueba de hipótesis para la correlación.\\
\item
  Ecuaciones lineales y modelo lineal simple.\\
\item
  Método de mínimos cuadrados.\\
\item
  Residuos y error estándar de la estimación.\\
\item
  Predicción e intervalos de confianza.\\
\item
  Coeficiente de determinación simple (R\^{}2).\\
\item
  Prueba de hipótesis sobre el parámetro de pendiente (\beta\_1).
\end{itemize}

En este laboratorio llevaremos varios de estos conceptos a la práctica
con \textbf{R}.

\section{Configuración inicial en
R}\label{configuraciuxf3n-inicial-en-r}

En esta sección cargaremos las \textbf{librerías} necesarias y
definiremos la \textbf{ruta a los datos}.

\subsection{Carga de librerías}\label{carga-de-libreruxedas}

\begin{Shaded}
\begin{Highlighting}[]
\CommentTok{\# Cargamos las librerías necesarias para el laboratorio}
\FunctionTok{library}\NormalTok{(openxlsx)  }\CommentTok{\# leer archivos Excel (.xlsx)}
\end{Highlighting}
\end{Shaded}

\begin{tcolorbox}[enhanced jigsaw, opacityback=0, left=2mm, rightrule=.15mm, arc=.35mm, colframe=quarto-callout-tip-color-frame, opacitybacktitle=0.6, toprule=.15mm, bottomtitle=1mm, toptitle=1mm, colbacktitle=quarto-callout-tip-color!10!white, colback=white, coltitle=black, leftrule=.75mm, breakable, titlerule=0mm, bottomrule=.15mm, title=\textcolor{quarto-callout-tip-color}{\faLightbulb}\hspace{0.5em}{Tip}]

Si alguna librería no está instalada, puedes hacerlo con:

\begin{Shaded}
\begin{Highlighting}[]
\FunctionTok{install.packages}\NormalTok{(}\StringTok{"openxlsx"}\NormalTok{)}
\end{Highlighting}
\end{Shaded}

\end{tcolorbox}

\subsection{Definir la ruta de
trabajo}\label{definir-la-ruta-de-trabajo}

Vamos a guardar la ruta donde están los datos en un objeto llamado
\texttt{ruta\_datos}.\\
Así solo modificamos una línea si cambiamos la carpeta en el futuro.

\begin{Shaded}
\begin{Highlighting}[]
\CommentTok{\# Definimos la ruta donde están los archivos de datos del laboratorio.}
\CommentTok{\# IMPORTANTE: Ajusta esta ruta si tu carpeta tiene otro nombre o ubicación.}

\NormalTok{ruta\_datos }\OtherTok{\textless{}{-}} \StringTok{"C:/Users/manue/Desktop/lab{-}econometria/labs\_epg/data\_epg"}

\CommentTok{\# Podemos verificar el contenido de la carpeta (opcional)}
\FunctionTok{list.files}\NormalTok{(ruta\_datos)}
\end{Highlighting}
\end{Shaded}

\begin{verbatim}
 [1] "annos_mantenimiento.xlsx" "auto_peso_consumo.xlsx"  
 [3] "costos.xlsx"              "data_PCA_Decathlon.csv"  
 [5] "data_PCA_ExpertWine.csv"  "Ejemplo1.xlsx"           
 [7] "Ejemplo2.xlsx"            "millaje.txt"             
 [9] "orange.csv"               "tabla_ejemplo_R.xlsx"    
\end{verbatim}

\begin{tcolorbox}[enhanced jigsaw, opacityback=0, left=2mm, rightrule=.15mm, arc=.35mm, colframe=quarto-callout-note-color-frame, opacitybacktitle=0.6, toprule=.15mm, bottomtitle=1mm, toptitle=1mm, colbacktitle=quarto-callout-note-color!10!white, colback=white, coltitle=black, leftrule=.75mm, breakable, titlerule=0mm, bottomrule=.15mm, title=\textcolor{quarto-callout-note-color}{\faInfo}\hspace{0.5em}{Nota}]

En R es recomendable usar \textbf{/} (slash) en lugar de
\textbf{\textbackslash{}} en las rutas de Windows.\\
Por eso escribimos \texttt{"C:/Users/manue/Desktop/..."} en lugar de
\texttt{"C:\textbackslash{}\textbackslash{}Users\textbackslash{}\textbackslash{}..."}.

\end{tcolorbox}

\section{Ejemplo 1: Correlación entre peso del auto y consumo de
gasolina}\label{ejemplo-1-correlaciuxf3n-entre-peso-del-auto-y-consumo-de-gasolina}

En este ejemplo estudiaremos la relación entre:

\begin{itemize}
\tightlist
\item
  \texttt{Peso\_Libras}: peso del automóvil (en libras).\\
\item
  \texttt{Consumo\_Millas\_por\_galon}: rendimiento (millas por galón).
\end{itemize}

La idea es:

\begin{enumerate}
\def\labelenumi{\arabic{enumi}.}
\tightlist
\item
  Graficar un \textbf{diagrama de dispersión}.\\
\item
  Calcular el \textbf{coeficiente de correlación}.\\
\item
  Realizar una \textbf{prueba de hipótesis} para ver si la correlación
  es distinta de cero.
\end{enumerate}

\subsection{Lectura de los datos de
autos}\label{lectura-de-los-datos-de-autos}

\begin{Shaded}
\begin{Highlighting}[]
\NormalTok{archivo\_autos }\OtherTok{\textless{}{-}} \FunctionTok{file.path}\NormalTok{(ruta\_datos, }\StringTok{"auto\_peso\_consumo.xlsx"}\NormalTok{)}

\NormalTok{datos }\OtherTok{\textless{}{-}} \FunctionTok{read.xlsx}\NormalTok{(}
\NormalTok{  archivo\_autos,}
  \AttributeTok{sheet    =} \StringTok{"Hoja1"}\NormalTok{,}
  \AttributeTok{colNames =} \ConstantTok{TRUE}
\NormalTok{)}

\CommentTok{\# Vemos las primeras filas}
\FunctionTok{head}\NormalTok{(datos)}
\end{Highlighting}
\end{Shaded}

\begin{verbatim}
  Auto Peso_Libras Consumo_Millas_por_galon
1    1        2743                     21.4
2    2        3518                     15.2
3    3        1855                     38.9
4    4        5214                     12.7
5    5        4341                     17.8
\end{verbatim}

Esperamos que el archivo contenga, al menos, las columnas:

\begin{itemize}
\tightlist
\item
  \texttt{Peso\_Libras}\strut \\
\item
  \texttt{Consumo\_Millas\_por\_galon}
\end{itemize}

\subsection{Diagrama de dispersión}\label{diagrama-de-dispersiuxf3n}

El \textbf{diagrama de dispersión} nos permite ver visualmente si existe
una relación lineal entre las variables.

\begin{Shaded}
\begin{Highlighting}[]
\FunctionTok{plot}\NormalTok{(}
\NormalTok{  datos}\SpecialCharTok{$}\NormalTok{Peso\_Libras,}
\NormalTok{  datos}\SpecialCharTok{$}\NormalTok{Consumo\_Millas\_por\_galon,}
  \AttributeTok{xlab =} \StringTok{"Peso (libras)"}\NormalTok{,}
  \AttributeTok{ylab =} \StringTok{"Consumo (millas por galón)"}\NormalTok{,}
  \AttributeTok{main =} \StringTok{"Relación entre peso del auto y consumo"}
\NormalTok{)}
\end{Highlighting}
\end{Shaded}

\begin{center}
\pandocbounded{\includegraphics[keepaspectratio]{lab02_epg_files/figure-pdf/ejemplo1-plot-1.pdf}}
\end{center}

\begin{tcolorbox}[enhanced jigsaw, opacityback=0, left=2mm, rightrule=.15mm, arc=.35mm, colframe=quarto-callout-note-color-frame, opacitybacktitle=0.6, toprule=.15mm, bottomtitle=1mm, toptitle=1mm, colbacktitle=quarto-callout-note-color!10!white, colback=white, coltitle=black, leftrule=.75mm, breakable, titlerule=0mm, bottomrule=.15mm, title=\textcolor{quarto-callout-note-color}{\faInfo}\hspace{0.5em}{Nota}]

\begin{itemize}
\tightlist
\item
  Si al aumentar el peso el consumo (millas por galón)
  \textbf{disminuye}, la nube de puntos tendrá una forma descendente →
  \textbf{correlación negativa}.\\
\item
  Si al aumentar el peso el consumo \textbf{aumentara}, veríamos una
  nube ascendente → \textbf{correlación positiva}.\\
\item
  Si no hay patrón claro, la correlación podría ser cercana a cero.
\end{itemize}

\end{tcolorbox}

\subsection{Cálculo de la
correlación}\label{cuxe1lculo-de-la-correlaciuxf3n}

El coeficiente de correlación de Pearson mide la \textbf{intensidad y
dirección} de la relación lineal entre dos variables numéricas.

\begin{Shaded}
\begin{Highlighting}[]
\NormalTok{r }\OtherTok{\textless{}{-}} \FunctionTok{cor}\NormalTok{(datos}\SpecialCharTok{$}\NormalTok{Peso\_Libras, datos}\SpecialCharTok{$}\NormalTok{Consumo\_Millas\_por\_galon)}
\NormalTok{r}
\end{Highlighting}
\end{Shaded}

\begin{verbatim}
[1] -0.8549912
\end{verbatim}

\begin{itemize}
\tightlist
\item
  (r) está entre -1 y 1.\\
\item
  (r \textless{} 0): relación negativa.\\
\item
  (r \textgreater{} 0): relación positiva.\\
\item
  (\textbar r\textbar) cercano a 1 → relación lineal fuerte.\\
\item
  (\textbar r\textbar) cercano a 0 → relación lineal débil.
\end{itemize}

\subsection{Prueba de hipótesis para la correlación (cálculo
manual)}\label{prueba-de-hipuxf3tesis-para-la-correlaciuxf3n-cuxe1lculo-manual}

En la teoría se plantea la prueba:

\[
H_0: \rho = 0 \quad \text{vs} \quad H_1: \rho \neq 0
\]

La idea es ver si la correlación poblacional (\rho) podría ser cero o
no.

En el script se calcula el \textbf{error estándar} del coeficiente de
correlación y luego el estadístico t:

\begin{Shaded}
\begin{Highlighting}[]
\CommentTok{\# Cálculo manual basado en la fórmula del error estándar de r}
\NormalTok{sr }\OtherTok{\textless{}{-}} \FunctionTok{sqrt}\NormalTok{((}\DecValTok{1} \SpecialCharTok{{-}}\NormalTok{ r) }\SpecialCharTok{/} \DecValTok{3}\NormalTok{)   }\CommentTok{\# comentario original: n número de datos menos 2}

\NormalTok{t }\OtherTok{\textless{}{-}}\NormalTok{ r }\SpecialCharTok{/}\NormalTok{ sr               }\CommentTok{\# estadístico t aproximado}

\NormalTok{t}
\end{Highlighting}
\end{Shaded}

\begin{verbatim}
[1] -1.087305
\end{verbatim}

Luego se calcula el \textbf{valor crítico} y el \textbf{p-valor} usando
la distribución t de Student:

\begin{Shaded}
\begin{Highlighting}[]
\NormalTok{c }\OtherTok{\textless{}{-}} \FunctionTok{qt}\NormalTok{(}\FloatTok{0.025}\NormalTok{, }\DecValTok{3}\NormalTok{, }\AttributeTok{lower.tail =} \ConstantTok{FALSE}\NormalTok{)  }\CommentTok{\# valor crítico (cola superior)}
\NormalTok{c}
\end{Highlighting}
\end{Shaded}

\begin{verbatim}
[1] 3.182446
\end{verbatim}

\begin{Shaded}
\begin{Highlighting}[]
\CommentTok{\# p{-}valor aproximado}
\FunctionTok{pt}\NormalTok{(}\SpecialCharTok{{-}}\NormalTok{t, }\DecValTok{3}\NormalTok{, }\AttributeTok{lower.tail =} \ConstantTok{FALSE}\NormalTok{)}
\end{Highlighting}
\end{Shaded}

\begin{verbatim}
[1] 0.1782267
\end{verbatim}

\begin{tcolorbox}[enhanced jigsaw, opacityback=0, left=2mm, rightrule=.15mm, arc=.35mm, colframe=quarto-callout-note-color-frame, opacitybacktitle=0.6, toprule=.15mm, bottomtitle=1mm, toptitle=1mm, colbacktitle=quarto-callout-note-color!10!white, colback=white, coltitle=black, leftrule=.75mm, breakable, titlerule=0mm, bottomrule=.15mm, title=\textcolor{quarto-callout-note-color}{\faInfo}\hspace{0.5em}{Nota}]

\begin{itemize}
\tightlist
\item
  Si el \textbf{p-valor} es pequeño (por ejemplo, menor que 0.05),
  rechazamos (H\_0) y concluimos que la correlación es
  \textbf{significativamente distinta de cero}.\\
\item
  Si el p-valor es grande, no tenemos evidencia suficiente para afirmar
  que exista correlación lineal distinta de cero.
\end{itemize}

\end{tcolorbox}

\subsection{\texorpdfstring{Prueba de hipótesis para la correlación con
\texttt{cor.test}}{Prueba de hipótesis para la correlación con cor.test}}\label{prueba-de-hipuxf3tesis-para-la-correlaciuxf3n-con-cor.test}

En lugar de hacer todos los cálculos ``a mano'', R nos ofrece la función
\texttt{cor.test}, que:

\begin{itemize}
\tightlist
\item
  Calcula el coeficiente de correlación.\\
\item
  Realiza la prueba de hipótesis.\\
\item
  Entrega el p-valor y un intervalo de confianza para (\rho).
\end{itemize}

\begin{Shaded}
\begin{Highlighting}[]
\FunctionTok{cor.test}\NormalTok{(datos}\SpecialCharTok{$}\NormalTok{Peso\_Libras, datos}\SpecialCharTok{$}\NormalTok{Consumo\_Millas\_por\_galon)}
\end{Highlighting}
\end{Shaded}

\begin{verbatim}

    Pearson's product-moment correlation

data:  datos$Peso_Libras and datos$Consumo_Millas_por_galon
t = -2.8553, df = 3, p-value = 0.06483
alternative hypothesis: true correlation is not equal to 0
95 percent confidence interval:
 -0.9902684  0.1110238
sample estimates:
       cor 
-0.8549912 
\end{verbatim}

\begin{tcolorbox}[enhanced jigsaw, opacityback=0, left=2mm, rightrule=.15mm, arc=.35mm, colframe=quarto-callout-tip-color-frame, opacitybacktitle=0.6, toprule=.15mm, bottomtitle=1mm, toptitle=1mm, colbacktitle=quarto-callout-tip-color!10!white, colback=white, coltitle=black, leftrule=.75mm, breakable, titlerule=0mm, bottomrule=.15mm, title=\textcolor{quarto-callout-tip-color}{\faLightbulb}\hspace{0.5em}{Tip}]

Siempre que sea posible, conviene \textbf{verificar los resultados
manuales} con funciones integradas como \texttt{cor.test}, ya que éstas
manejan bien detalles como el tamaño de muestra, grados de libertad y
supuestos.

\end{tcolorbox}

\section{Ejemplo 2: Correlación y regresión del costo de
mantenimiento}\label{ejemplo-2-correlaciuxf3n-y-regresiuxf3n-del-costo-de-mantenimiento}

En este ejemplo utilizamos datos de:

\begin{itemize}
\tightlist
\item
  \texttt{Tiempo\_operacion}: años de operación de un bus.\\
\item
  \texttt{Costo\_Mantenimiento}: costo anual de mantenimiento (por
  ejemplo, en dólares).
\end{itemize}

Queremos:

\begin{enumerate}
\def\labelenumi{\arabic{enumi}.}
\tightlist
\item
  Ver si existe correlación entre el tiempo de operación y el costo de
  mantenimiento.\\
\item
  Ajustar una \textbf{regresión lineal simple} para predecir el costo a
  partir del tiempo.\\
\item
  Evaluar los residuos y la calidad del ajuste.\\
\item
  Calcular predicciones e intervalos de confianza.
\end{enumerate}

\subsection{Lectura de los datos de
mantenimiento}\label{lectura-de-los-datos-de-mantenimiento}

\begin{Shaded}
\begin{Highlighting}[]
\NormalTok{archivo\_mant }\OtherTok{\textless{}{-}} \FunctionTok{file.path}\NormalTok{(ruta\_datos, }\StringTok{"annos\_mantenimiento.xlsx"}\NormalTok{)}

\NormalTok{datos2 }\OtherTok{\textless{}{-}} \FunctionTok{read.xlsx}\NormalTok{(}
\NormalTok{  archivo\_mant,}
  \AttributeTok{sheet    =} \StringTok{"Hoja1"}\NormalTok{,}
  \AttributeTok{colNames =} \ConstantTok{TRUE}
\NormalTok{)}

\FunctionTok{head}\NormalTok{(datos2)}
\end{Highlighting}
\end{Shaded}

\begin{verbatim}
  Bus Costo_Mantenimiento Tiempo_operacion
1   1                 859                8
2   2                 682                5
3   3                 471                3
4   4                 708                9
5   5                1094               11
6   6                 224                2
\end{verbatim}

Esperamos las columnas:

\begin{itemize}
\tightlist
\item
  \texttt{Tiempo\_operacion}\strut \\
\item
  \texttt{Costo\_Mantenimiento}
\end{itemize}

\subsection{Diagrama de dispersión}\label{diagrama-de-dispersiuxf3n-1}

\begin{Shaded}
\begin{Highlighting}[]
\FunctionTok{plot}\NormalTok{(}
\NormalTok{  datos2}\SpecialCharTok{$}\NormalTok{Tiempo\_operacion,}
\NormalTok{  datos2}\SpecialCharTok{$}\NormalTok{Costo\_Mantenimiento,}
  \AttributeTok{xlab =} \StringTok{"Tiempo de operación (años)"}\NormalTok{,}
  \AttributeTok{ylab =} \StringTok{"Costo de mantenimiento (unidades monetarias)"}\NormalTok{,}
  \AttributeTok{main =} \StringTok{"Relación entre tiempo de operación y costo de mantenimiento"}
\NormalTok{)}
\end{Highlighting}
\end{Shaded}

\begin{center}
\pandocbounded{\includegraphics[keepaspectratio]{lab02_epg_files/figure-pdf/ejemplo2-plot-1.pdf}}
\end{center}

\begin{tcolorbox}[enhanced jigsaw, opacityback=0, left=2mm, rightrule=.15mm, arc=.35mm, colframe=quarto-callout-note-color-frame, opacitybacktitle=0.6, toprule=.15mm, bottomtitle=1mm, toptitle=1mm, colbacktitle=quarto-callout-note-color!10!white, colback=white, coltitle=black, leftrule=.75mm, breakable, titlerule=0mm, bottomrule=.15mm, title=\textcolor{quarto-callout-note-color}{\faInfo}\hspace{0.5em}{Nota}]

Este gráfico permite ver si al aumentar los años de operación los costos
de mantenimiento tienden a subir.\\
Si la nube de puntos sugiere una recta ascendente, tiene sentido ajustar
un modelo lineal.

\end{tcolorbox}

\subsection{Cálculo de la correlación y prueba de
hipótesis}\label{cuxe1lculo-de-la-correlaciuxf3n-y-prueba-de-hipuxf3tesis}

\begin{Shaded}
\begin{Highlighting}[]
\NormalTok{r }\OtherTok{\textless{}{-}} \FunctionTok{cor}\NormalTok{(datos2}\SpecialCharTok{$}\NormalTok{Tiempo\_operacion, datos2}\SpecialCharTok{$}\NormalTok{Costo\_Mantenimiento)}
\NormalTok{r}
\end{Highlighting}
\end{Shaded}

\begin{verbatim}
[1] 0.9376733
\end{verbatim}

Nuevamente, calculamos el error estándar y el estadístico t de forma
manual (siguiendo la lógica del script original):

\begin{Shaded}
\begin{Highlighting}[]
\NormalTok{sr }\OtherTok{\textless{}{-}} \FunctionTok{sqrt}\NormalTok{((}\DecValTok{1} \SpecialCharTok{{-}}\NormalTok{ r) }\SpecialCharTok{/} \DecValTok{7}\NormalTok{)  }\CommentTok{\# comentario original: aquí se usa 7 como "n {-} 2"}

\NormalTok{t }\OtherTok{\textless{}{-}}\NormalTok{ r }\SpecialCharTok{/}\NormalTok{ sr}
\NormalTok{t}
\end{Highlighting}
\end{Shaded}

\begin{verbatim}
[1] 9.937184
\end{verbatim}

Se podría obtener un valor crítico (aunque en el script se reutiliza un
valor con 3 grados de libertad), y luego:

\begin{Shaded}
\begin{Highlighting}[]
\FunctionTok{cor.test}\NormalTok{(datos2}\SpecialCharTok{$}\NormalTok{Tiempo\_operacion, datos2}\SpecialCharTok{$}\NormalTok{Costo\_Mantenimiento)}
\end{Highlighting}
\end{Shaded}

\begin{verbatim}

    Pearson's product-moment correlation

data:  datos2$Tiempo_operacion and datos2$Costo_Mantenimiento
t = 7.1388, df = 7, p-value = 0.0001872
alternative hypothesis: true correlation is not equal to 0
95 percent confidence interval:
 0.7250800 0.9870994
sample estimates:
      cor 
0.9376733 
\end{verbatim}

\begin{tcolorbox}[enhanced jigsaw, opacityback=0, left=2mm, rightrule=.15mm, arc=.35mm, colframe=quarto-callout-tip-color-frame, opacitybacktitle=0.6, toprule=.15mm, bottomtitle=1mm, toptitle=1mm, colbacktitle=quarto-callout-tip-color!10!white, colback=white, coltitle=black, leftrule=.75mm, breakable, titlerule=0mm, bottomrule=.15mm, title=\textcolor{quarto-callout-tip-color}{\faLightbulb}\hspace{0.5em}{Tip}]

\texttt{cor.test} es la forma recomendada de hacer la prueba de
hipótesis para la correlación, ya que usa la fórmula teórica correcta y
ajusta automáticamente los grados de libertad.

\end{tcolorbox}

\subsection{Ajuste del modelo de regresión lineal
simple}\label{ajuste-del-modelo-de-regresiuxf3n-lineal-simple}

Planteamos el modelo:

\[\text{Costo\_Mantenimiento} = \beta\_0 + \beta\_1
\cdot \text{Tiempo\_operacion} + \varepsilon \]

Lo estimamos con \texttt{lm}:

\begin{Shaded}
\begin{Highlighting}[]
\NormalTok{modelo }\OtherTok{\textless{}{-}} \FunctionTok{lm}\NormalTok{(Costo\_Mantenimiento }\SpecialCharTok{\textasciitilde{}}\NormalTok{ Tiempo\_operacion, }\AttributeTok{data =}\NormalTok{ datos2)}

\FunctionTok{summary}\NormalTok{(modelo)}
\end{Highlighting}
\end{Shaded}

\begin{verbatim}

Call:
lm(formula = Costo_Mantenimiento ~ Tiempo_operacion, data = datos2)

Residuals:
    Min      1Q  Median      3Q     Max 
-138.47 -124.55   40.88   83.45  119.21 

Coefficients:
                 Estimate Std. Error t value Pr(>|t|)    
(Intercept)       208.203     75.002   2.776 0.027457 *  
Tiempo_operacion   70.918      9.934   7.139 0.000187 ***
---
Signif. codes:  0 '***' 0.001 '**' 0.01 '*' 0.05 '.' 0.1 ' ' 1

Residual standard error: 111.6 on 7 degrees of freedom
Multiple R-squared:  0.8792,    Adjusted R-squared:  0.862 
F-statistic: 50.96 on 1 and 7 DF,  p-value: 0.0001872
\end{verbatim}

El output de \texttt{summary(modelo)} incluye:

\begin{itemize}
\tightlist
\item
  Estimaciones de (\beta\_0) (intercepto) y (\beta\_1) (pendiente).\\
\item
  Error estándar de cada coeficiente.\\
\item
  Estadístico t y p-valor para probar si los coeficientes son distintos
  de cero.\\
\item
  (R\^{}2): porcentaje de variabilidad en el costo explicado por el
  tiempo de operación.
\end{itemize}

\begin{tcolorbox}[enhanced jigsaw, opacityback=0, left=2mm, rightrule=.15mm, arc=.35mm, colframe=quarto-callout-note-color-frame, opacitybacktitle=0.6, toprule=.15mm, bottomtitle=1mm, toptitle=1mm, colbacktitle=quarto-callout-note-color!10!white, colback=white, coltitle=black, leftrule=.75mm, breakable, titlerule=0mm, bottomrule=.15mm, title=\textcolor{quarto-callout-note-color}{\faInfo}\hspace{0.5em}{Nota}]

\begin{itemize}
\tightlist
\item
  Si el p-valor asociado a la pendiente (\beta\_1) es pequeño (ej.
  \textless{} 0.05), concluimos que el tiempo de operación es un
  \textbf{buen predictor} del costo de mantenimiento.\\
\item
  Un (R\^{}2) alto indica que el modelo lineal explica gran parte de la
  variabilidad del costo.
\end{itemize}

\end{tcolorbox}

\subsection{Predicción para 5 años de
operación}\label{predicciuxf3n-para-5-auxf1os-de-operaciuxf3n}

Supongamos que queremos predecir el \textbf{costo de mantenimiento} para
un bus con \textbf{5 años} de operación.

\begin{Shaded}
\begin{Highlighting}[]
\NormalTok{nuevo }\OtherTok{\textless{}{-}} \FunctionTok{data.frame}\NormalTok{(}\AttributeTok{Tiempo\_operacion =} \FunctionTok{c}\NormalTok{(}\DecValTok{5}\NormalTok{))  }\CommentTok{\# valor donde evaluamos el modelo}

\NormalTok{valor\_predicho }\OtherTok{\textless{}{-}} \FunctionTok{predict}\NormalTok{(}\AttributeTok{object =}\NormalTok{ modelo, }\AttributeTok{newdata =}\NormalTok{ nuevo)}

\NormalTok{valor\_predicho}
\end{Highlighting}
\end{Shaded}

\begin{verbatim}
      1 
562.794 
\end{verbatim}

Este es el \textbf{valor esperado} de costo de mantenimiento según el
modelo lineal.

\subsection{Análisis de residuos}\label{anuxe1lisis-de-residuos}

Los residuos son las diferencias entre los valores observados y los
valores ajustados por el modelo:

\[ \hat{\varepsilon}\_i = y_i - \hat{y}\_i \]

\begin{Shaded}
\begin{Highlighting}[]
\CommentTok{\# Vector de residuos}
\NormalTok{modelo}\SpecialCharTok{$}\NormalTok{residuals}
\end{Highlighting}
\end{Shaded}

\begin{verbatim}
         1          2          3          4          5          6          7 
  83.45158  119.20599   50.04225 -138.46655  105.69718 -126.03961   40.87852 
         8          9 
-124.54842  -10.22095 
\end{verbatim}

\begin{Shaded}
\begin{Highlighting}[]
\CommentTok{\# Histograma de residuos}
\FunctionTok{hist}\NormalTok{(}
\NormalTok{  modelo}\SpecialCharTok{$}\NormalTok{residuals,}
  \AttributeTok{main =} \StringTok{"Histograma de residuos"}\NormalTok{,}
  \AttributeTok{xlab =} \StringTok{"Residuo"}
\NormalTok{)}
\end{Highlighting}
\end{Shaded}

\begin{center}
\pandocbounded{\includegraphics[keepaspectratio]{lab02_epg_files/figure-pdf/ejemplo2-residuos-1.pdf}}
\end{center}

\begin{Shaded}
\begin{Highlighting}[]
\CommentTok{\# Densidad de los residuos}
\FunctionTok{plot}\NormalTok{(}
  \FunctionTok{density}\NormalTok{(modelo}\SpecialCharTok{$}\NormalTok{residuals),}
  \AttributeTok{main =} \StringTok{"Densidad de residuos"}\NormalTok{,}
  \AttributeTok{xlab =} \StringTok{"Residuo"}
\NormalTok{)}
\end{Highlighting}
\end{Shaded}

\begin{center}
\pandocbounded{\includegraphics[keepaspectratio]{lab02_epg_files/figure-pdf/ejemplo2-residuos-2.pdf}}
\end{center}

\begin{Shaded}
\begin{Highlighting}[]
\CommentTok{\# Media de los residuos}
\FunctionTok{mean}\NormalTok{(modelo}\SpecialCharTok{$}\NormalTok{residuals)}
\end{Highlighting}
\end{Shaded}

\begin{verbatim}
[1] -2.368187e-15
\end{verbatim}

\begin{tcolorbox}[enhanced jigsaw, opacityback=0, left=2mm, rightrule=.15mm, arc=.35mm, colframe=quarto-callout-note-color-frame, opacitybacktitle=0.6, toprule=.15mm, bottomtitle=1mm, toptitle=1mm, colbacktitle=quarto-callout-note-color!10!white, colback=white, coltitle=black, leftrule=.75mm, breakable, titlerule=0mm, bottomrule=.15mm, title=\textcolor{quarto-callout-note-color}{\faInfo}\hspace{0.5em}{Nota}]

\begin{itemize}
\tightlist
\item
  En un buen modelo lineal, los residuos deberían:

  \begin{itemize}
  \tightlist
  \item
    Tener \textbf{media cercana a cero}.\\
  \item
    No mostrar patrones sistemáticos.\\
  \item
    Aproximarse a una \textbf{distribución normal} (especialmente
    importante para los intervalos de confianza).\\
  \end{itemize}
\item
  El histograma y la densidad ayudan a evaluar visualmente estas
  propiedades.
\end{itemize}

\end{tcolorbox}

\subsection{Cálculo del error estándar de la estimación
(s\_\{y,x\})}\label{cuxe1lculo-del-error-estuxe1ndar-de-la-estimaciuxf3n-s_yx}

El script calcula manualmente el error estándar de la estimación a
partir de los residuos:

\[ s\_{y,x} = \sqrt{\frac{\sum \hat{\varepsilon}_i^2}{n - 2}} \]

\begin{Shaded}
\begin{Highlighting}[]
\NormalTok{s\_yx }\OtherTok{\textless{}{-}} \FunctionTok{sqrt}\NormalTok{(}\FunctionTok{sum}\NormalTok{(modelo}\SpecialCharTok{$}\NormalTok{residuals}\SpecialCharTok{\^{}}\DecValTok{2}\NormalTok{) }\SpecialCharTok{/} \DecValTok{7}\NormalTok{)  }\CommentTok{\# aquí se usa 7 como "n {-} 2"}
\NormalTok{s\_yx}
\end{Highlighting}
\end{Shaded}

\begin{verbatim}
[1] 111.6097
\end{verbatim}

Este valor mide la \textbf{dispersión típica} de los puntos alrededor de
la recta de regresión.

\subsection{Intervalo de predicción para un valor individual (x\_0 =
5)}\label{intervalo-de-predicciuxf3n-para-un-valor-individual-x_0-5}

Para un valor específico (x\_0 = 5), el error estándar de la predicción
se calcula (según la teoría) como:

\[ s\_{\hat{y}x} = s\_{y,x}
\sqrt{1 + \frac{1}{n} + \frac{(x_0 - \bar{x})^2}{\sum (x_i - \bar{x})^2}}
\]

El script implementa esto en varios pasos.

\begin{Shaded}
\begin{Highlighting}[]
\NormalTok{raiz }\OtherTok{\textless{}{-}} \FunctionTok{sqrt}\NormalTok{(}
  \DecValTok{1} \SpecialCharTok{+}
    \DecValTok{1} \SpecialCharTok{/} \DecValTok{9} \SpecialCharTok{+}
\NormalTok{    (}\DecValTok{5} \SpecialCharTok{{-}} \FunctionTok{mean}\NormalTok{(datos2}\SpecialCharTok{$}\NormalTok{Tiempo\_operacion))}\SpecialCharTok{\^{}}\DecValTok{2} \SpecialCharTok{/}
      \FunctionTok{sum}\NormalTok{((datos2}\SpecialCharTok{$}\NormalTok{Tiempo\_operacion }\SpecialCharTok{{-}} \FunctionTok{mean}\NormalTok{(datos2}\SpecialCharTok{$}\NormalTok{Tiempo\_operacion))}\SpecialCharTok{\^{}}\DecValTok{2}\NormalTok{)}
\NormalTok{)}

\NormalTok{s\_hatyx }\OtherTok{\textless{}{-}}\NormalTok{ s\_yx }\SpecialCharTok{*}\NormalTok{ raiz}
\NormalTok{s\_hatyx}
\end{Highlighting}
\end{Shaded}

\begin{verbatim}
[1] 118.6576
\end{verbatim}

Luego se obtiene el valor crítico t y se calculan los límites del
\textbf{intervalo de predicción}:

\begin{Shaded}
\begin{Highlighting}[]
\NormalTok{t\_crit }\OtherTok{\textless{}{-}} \FunctionTok{qt}\NormalTok{(}\FloatTok{0.025}\NormalTok{, }\DecValTok{7}\NormalTok{, }\AttributeTok{lower.tail =} \ConstantTok{FALSE}\NormalTok{)  }\CommentTok{\# valor crítico t para 95\%}

\NormalTok{lim\_sup }\OtherTok{\textless{}{-}}\NormalTok{ valor\_predicho }\SpecialCharTok{+}\NormalTok{ t\_crit }\SpecialCharTok{*}\NormalTok{ s\_hatyx}
\NormalTok{lim\_inf }\OtherTok{\textless{}{-}}\NormalTok{ valor\_predicho }\SpecialCharTok{{-}}\NormalTok{ t\_crit }\SpecialCharTok{*}\NormalTok{ s\_hatyx}

\NormalTok{lim\_inf}
\end{Highlighting}
\end{Shaded}

\begin{verbatim}
       1 
282.2134 
\end{verbatim}

\begin{Shaded}
\begin{Highlighting}[]
\NormalTok{lim\_sup}
\end{Highlighting}
\end{Shaded}

\begin{verbatim}
       1 
843.3746 
\end{verbatim}

\begin{tcolorbox}[enhanced jigsaw, opacityback=0, left=2mm, rightrule=.15mm, arc=.35mm, colframe=quarto-callout-note-color-frame, opacitybacktitle=0.6, toprule=.15mm, bottomtitle=1mm, toptitle=1mm, colbacktitle=quarto-callout-note-color!10!white, colback=white, coltitle=black, leftrule=.75mm, breakable, titlerule=0mm, bottomrule=.15mm, title=\textcolor{quarto-callout-note-color}{\faInfo}\hspace{0.5em}{Nota}]

\begin{itemize}
\tightlist
\item
  Este intervalo responde a la pregunta:\\
  \textgreater{} ``¿En qué rango esperamos que caiga \textbf{un nuevo
  costo de mantenimiento individual} para un bus con 5 años de
  operación, con un 95\% de confianza?''\\
\item
  Es más ancho que el intervalo para la \textbf{media} porque incluye la
  variabilidad individual.
\end{itemize}

\end{tcolorbox}

\subsection{Intervalo de confianza para la media del costo cuando (x\_0
= 5)}\label{intervalo-de-confianza-para-la-media-del-costo-cuando-x_0-5}

Si en lugar de un valor individual queremos estimar la \textbf{media
poblacional} del costo para buses con 5 años de operación, el error
estándar es:

{[} s\_\{\hat{\mu}x\} = s\_\{y,x\}
\sqrt{\frac{1}{n} + \frac{(x_0 - \bar{x})^2}{\sum (x_i - \bar{x})^2}}
{]}

En el script:

\begin{Shaded}
\begin{Highlighting}[]
\NormalTok{raiz1 }\OtherTok{\textless{}{-}} \FunctionTok{sqrt}\NormalTok{(}
  \DecValTok{1} \SpecialCharTok{/} \DecValTok{9} \SpecialCharTok{+}
\NormalTok{    (}\DecValTok{5} \SpecialCharTok{{-}} \FunctionTok{mean}\NormalTok{(datos2}\SpecialCharTok{$}\NormalTok{Tiempo\_operacion))}\SpecialCharTok{\^{}}\DecValTok{2} \SpecialCharTok{/}
      \FunctionTok{sum}\NormalTok{((datos2}\SpecialCharTok{$}\NormalTok{Tiempo\_operacion }\SpecialCharTok{{-}} \FunctionTok{mean}\NormalTok{(datos2}\SpecialCharTok{$}\NormalTok{Tiempo\_operacion))}\SpecialCharTok{\^{}}\DecValTok{2}\NormalTok{)}
\NormalTok{)}

\NormalTok{s\_hatmux }\OtherTok{\textless{}{-}}\NormalTok{ s\_yx }\SpecialCharTok{*}\NormalTok{ raiz1}
\NormalTok{s\_hatmux}
\end{Highlighting}
\end{Shaded}

\begin{verbatim}
[1] 40.28504
\end{verbatim}

Sin embargo, una forma más directa es usar \texttt{predict} con
\texttt{interval\ =\ "confidence"}:

\begin{Shaded}
\begin{Highlighting}[]
\NormalTok{valor\_predicho\_conf }\OtherTok{\textless{}{-}} \FunctionTok{predict}\NormalTok{(}
  \AttributeTok{object =}\NormalTok{ modelo,}
  \AttributeTok{newdata =}\NormalTok{ nuevo,}
  \AttributeTok{interval =} \StringTok{"confidence"}  \CommentTok{\# intervalo de confianza para la media}
\NormalTok{)}

\NormalTok{valor\_predicho\_conf}
\end{Highlighting}
\end{Shaded}

\begin{verbatim}
      fit     lwr     upr
1 562.794 467.535 658.053
\end{verbatim}

\begin{tcolorbox}[enhanced jigsaw, opacityback=0, left=2mm, rightrule=.15mm, arc=.35mm, colframe=quarto-callout-tip-color-frame, opacitybacktitle=0.6, toprule=.15mm, bottomtitle=1mm, toptitle=1mm, colbacktitle=quarto-callout-tip-color!10!white, colback=white, coltitle=black, leftrule=.75mm, breakable, titlerule=0mm, bottomrule=.15mm, title=\textcolor{quarto-callout-tip-color}{\faLightbulb}\hspace{0.5em}{Tip}]

\begin{itemize}
\tightlist
\item
  \texttt{interval\ =\ "confidence"}: intervalo de confianza para la
  \textbf{media} del costo en (x\_0).\\
\item
  \texttt{interval\ =\ "prediction"}: intervalo de predicción para un
  \textbf{nuevo valor individual}.
\end{itemize}

\end{tcolorbox}

\section{Resumen del laboratorio}\label{resumen-del-laboratorio}

En este laboratorio aprendiste a:

\begin{itemize}
\tightlist
\item
  Construir y analizar \textbf{diagramas de dispersión}.\\
\item
  Calcular el \textbf{coeficiente de correlación} y probar si es
  significativamente distinto de cero.\\
\item
  Ajustar un \textbf{modelo de regresión lineal simple} en R con
  \texttt{lm}.\\
\item
  Interpretar el output de \texttt{summary(modelo)} (coeficientes,
  p-valores, (R\^{}2)).\\
\item
  Analizar los \textbf{residuos} para evaluar los supuestos del
  modelo.\\
\item
  Calcular \textbf{predicciones puntuales}.\\
\item
  Construir \textbf{intervalos de predicción} e \textbf{intervalos de
  confianza} usando tanto fórmulas como la función \texttt{predict}.
\end{itemize}

Estos conceptos son fundamentales para los análisis econométricos que
verás más adelante, donde la regresión lineal es una herramienta
central.




\end{document}
